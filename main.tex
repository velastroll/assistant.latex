\documentclass[openright,twoside,10pt]{book}
\usepackage[b5paper,left=2cm,top=2.5cm,right=1.5cm,bottom=2.5cm]{geometry} 
\usepackage[spanish]{babel} % espanol
\usepackage[utf8]{inputenc} % acentos sin codigo
\usepackage{graphicx} % gráficos
\usepackage{lscape}
\usepackage{fancyvrb}
\usepackage{fancyhdr}
\usepackage{wrapfig}
\setlength{\parskip}{10pt plus 1pt minus 1pt}
 % aqui definimos el encabezado de las paginas pares e impares.
\rhead[]{}

\renewcommand{\headrulewidth}{0.5pt}

% aqui definimos el pie de pagina de las paginas pares e impares.
\rfoot[\thepage]{\thepage}
\cfoot[]{}
\renewcommand{\footrulewidth}{0pt}

%redefino el verbatim
%\renewenvironment{verbatim}{\begin{Verbatim}[frame=single,fontsize=\small]}{\end{Verbatim}}


% aqui definimos el encabezado y pie de pagina de la pagina inicial de un capitulo.
\fancypagestyle{plain}{
\fancyhead[R]{}
\fancyfoot[C]{}
\fancyfoot[R]{\thepage}
\renewcommand{\headrulewidth}{0.5pt}
\renewcommand{\footrulewidth}{0pt}
}

\pagestyle{fancy} % seleccionamos un estilo



\date{16 de marzo de 2016}
\author{Tú}

\title{TFG: Plantilla}

\begin{document}

\begin{titlepage}

\begin{center}
\vspace*{-1in}
\begin{figure}[htb]
\begin{center}
\includegraphics[width=3cm]{./img/logo}
\end{center}
\end{figure}
\begin{large}
\textbf{Universidad de Valladolid}
\end{large}

\vspace*{0.15in}

\vspace*{0.6in}
\begin{large}
\textbf{ESCUELA DE INGENIERÍA INFORMÁTICA}

\end{large}
\vspace*{0.2in}
\textbf{ GRADO EN INGENIERÍA INFORMÁTICA}\\
\textbf{ MENCIÓN EN *algo, no?* }
\vspace*{0.5in}
\rule{140mm}{0.1mm}\\
\vspace*{0.3in}
\begin{large}
\textbf{{\LARGE Fantástica plantilla para hacer TFGs con \LaTeX\\}}
\end{large}
\vspace*{0.6in}
\rule{140mm}{0.1mm}\\
\vspace*{2in}
\begin{large}
\begin{flushright}
\textbf{Alumno/a: Alguien \\
\vspace*{0.3in}
Tutor/es/as: }
\end{flushright}
\end{large}
\end{center}

\end{titlepage}

\newpage
\mbox{}	
\thispagestyle{empty} % para que no se numere esta página

\chapter*{}
\pagenumbering{Roman} % para comenzar la numeración de paginas en números romanos
\begin{flushright}
\textit{%Dedicatoria,\\
Pequeña dedicatoria.}
\end{flushright}

\chapter*{Agradecimientos} % si no queremos que añada la palabra "Capitulo"
\addcontentsline{toc}{chapter}{Agradecimientos} % si queremos que aparezca en el índice
\markboth{AGRADECIMIENTOS}{AGRADECIMIENTOS} % encabezado 

% Aquí agradecer

\chapter*{Resumen} % si no queremos que añada la palabra "Capitulo"
\addcontentsline{toc}{chapter}{Resumen} % si queremos que aparezca en el índice
\markboth{RESUMEN}{RESUMEN} % encabezado
\begin{flushleft}

% Aquí el resumen

\end{flushleft}


\chapter*{Abstract} % si no queremos que añada la palabra "Capitulo"
\addcontentsline{toc}{chapter}{Abstract} % si queremos que aparezca en el índice
\markboth{ABSTRACT}{ABSTRACT} % encabezado
\begin{flushleft}

% Aquí va el abstract

\end{flushleft}

\tableofcontents % indice de contenidos

\cleardoublepage
\addcontentsline{toc}{chapter}{Lista de figuras} % para que aparezca en el indice de contenidos
\listoffigures % indice de figuras

\cleardoublepage
\addcontentsline{toc}{chapter}{Lista de tablas} % para que aparezca en el indice de contenidos
\listoftables % indice de tablas

\chapter{Introducción}\label{cap.introduccion}
\pagenumbering{arabic} % para empezar la numeración con números
\input{./secciones/s0idea.tex}

\chapter{Estado del arte}\label{cap.arte}
\input{./secciones/s1arte.tex}

\chapter{Líneas futuras}\label{cap.futuro}
\section{Introducción}
\subsection{Una subsección}
 


\chapter{Conclusiones}\label{cap.conclusiones}
\input{./secciones/s5concluyo.tex}

\appendix

\chapter{Manual de usuario}\label{aped.A}
\input{./anexos/a1manual.tex}
\chapter{Contenidos del CD-ROM}\label{aped.B}
\input{./anexos/a2cdrom.tex}

%Esto es una cita: \cite{ej}. Tiene que hacer referencia a la etiqueta de un bibitem.

%Esto es un enlace \href{www.enlace.net}{Enlace}


\cleardoublepage
\addcontentsline{toc}{chapter}{Bibliografía}
%\renewcommand\bibname{Referencias Web}
\begin{thebibliography}{X}

\bibitem{ref1} \textit{Ejemplo}, \\
\textsc{ejemplo.com}.
\\Recuperado a tal fecha, \\de \href{http://ejemplo.com}


\end{thebibliography}


\end{document}
