\label{planificacion}

\

Una vez conocido el método de trabajo que se seguirá para el desarrollo del proyecto, se deben conocer los plazos en los cuales el proyecto debe ser finalizado y entregado.

La fecha de inicio del proyecto es, por tanto, \textit{el 15 de Octubre de 2019}, y la fecha límite en la cual el proyecto debe ser entregado data del \textit{29 de Febrero de 2020}.

\subsubsection{Disponibilidad del desarrollador}

El proyecto se debe ajustar a una extensión de 300 horas, como fué especificado en el cálculo de costes en el apartado \ref{duracion_proyecto} del presente documento.

En cuanto a la disponibilidad del proyecto, el desarrollador carece de suficiente tiempo  al día como para elaborar 8 horas díarias, asemejándose a un trabajo real, debido a tener que compaginar las 7 horas diarias de un trabajo externo, con las necesarias para la completitud de dos asignaturas de la Universidad, y el presente proyecto.

Tras una planificación semanal, el desarrollador es capaz de asignar una media de 18.75 horas semanales, lo que supondría poder finalizar el proyecto en:

\begin{center}
    \textit{18.75horas/semana / 7 dias/semana = 2,67horas/dia }
    
    \textit{300horas / 2,67horas/dia = 112,5 días }  

\end{center}

\subsubsection{Organización de prototipos}

De este modo tras los cálculos anteriores, el proyecto sería finalizado aproximadamente 113 días después del inicio de este, datando de una fecha de finalización prevista para el día \textit{4 de Febrero de 2020}, teniendo una ventana para el posible retraso del proyecto de 25 días.

En cuanto a la planificación de las tareas, se contempla la posibilidad de entregar prototipos al usuario cada dos semanas, de este modo la planificación final constará de: 

    \begin{center}
    \textit{300horas/TFG / 18,75horas/semana = 16 semanas/TFG }
    
    \textit{ 16semanas / 2semanas/ciclo = 8 ciclos}
    \end{center}{}

Como consecuencia, el proyecto deberá ser realizado tras la iteración de 8 ciclos, obteniendo un prototipo en la finalización de cada iteración, resultando un total de 8 prototipos.
