\section{Introducción}

En este proyecto se abordará el proceso de control y monitorización remota de dispositivos asistentes inteligentes.

Tras la lectura de este documento se comprenderán tanto los motivos por los cuales se ha decidido tomar esta opción, como el proceso de despliegue del sistema, pasando por las fases de implementación en las que se enseñará a replicar proyectos de estructura similiar, como por las fases de desarrollo en la cuales se estudian los posibles riesgos y características del proyecto, al igual que por el plan de desarrollo en el cual se programa toda la elaboración.

\section{Motivación}

Los asistentes inteligentes están en auge y las grandes empresas están dando acceso a estos servicios de manera gratuita, donde lo único que hay que hacer para disfrutar de uno de ellos es pagar es el dispositivo físico, asumiendo únicamente los costes del hardware, lo que hace cuestionar la idea del modelo de negocio que están siguiendo para que salga rentable.

El desarrollo de un sistema software capaz de dar respuesta a cada cuestión que un usuario se llegue a plantear, al igual que las infraestructuras que soporten todo un sistema capaz de retroalimentarse mediante la información que le proporciona el usuario tiene un coste muy elevado como para ofrecer ese servicio de manera gratuíta. Un modelo de negocio en el cual  se ofrece un dispositivo a un bajo precio con unas prestaciones de gran nivel sin cobrar a mayores una cuota mensual da que pensar, por ejemplo, que el producto mediante el cual está basado su modelo de negocio no es el que el usuario cree: Para el usuario, el producto es el dispositivo que le va a servir un uso, pero para la empresa, el producto es toda la información personal que le va a verter cada usuario que lo utilice a través del dispositivo. La información recopilada a través de los asistentes, al ser obtenida directamente de los hogares de cada usuario no es solamente privada, si no que también es íntima, lo que hace cuestionarse si realmente la comunidad de usuarios sabe que hay empresas aprovechándose de las conversaciones de su día a día para recolectar toda la información de un usuario que fluye en los lugares más personales de cada hogar, capturando información con gran potencial, ya sea sobre gustos, tendencias, necesidades o inclinaciones políticas, con el fin de poder no solo utilizar esos datos, sino también pudiendo incluso venderlos a terceros, a través de los cuales pueden llegar a intentar manipularnos.

Esta ignorancia global sobre el tratamiento de los datos obtenidos por el dispositivo hace replantear la posibilidad de creacción de un nuevo dispositivo asistente, el cual haga más facil la vida de un sesgo de la sociedad al cual iría orientado, sin tener la necesidad ni posibilidad de almacenar datos de carácter sensible.

Por otro lado también se observa otra gran motivación para la elaboración del proyecto, y es producida por la migración por la parte joven de la sociedad en los tiempos que acontecen actualmente, lo cual está provocando una despobación en sus lugares de origen, dejando a los familiares de mayor edad en soledad, siendo la parte de la familia que a grosso modo necesita más atención y requiere más ayuda para pasar el día.

La implementación de un asistente inteligente el cual ayude a este conjunto de la sociedad a entretenerse proponiendo tanto eventos cercanos a ellos, como respondiendo sus dudas de una manera rápida, o sirviéndoles para buscar ayuda en caso de posible caída o solicitud de auxilio, puede ser una gran herramienta que mejore su calidad de vida.


\section{Objetivos}

El objetivo de este proyecto es, por tanto, el despliegue de un sistema que permita la conexión de dispositivos asistentes al servidor de manera automática y remota, pudiendo ser controlados desde una web de administración, al igual que pudiendo ver estadísticas de uso para comprobar el estado de los usuarios. 

Estos dispositivos estarán orientados a personas de la tercera edad, estando cada dispositivo asignado a una persona, de la cual se conoce su localidad con el fin de obtener información relacionada a la hora de responder.

Los dispositivos podrán entonces ser controlados remotamente por parte de los administradores y podrán ser configurados tanto de manera individual, como en función de su localidad, o de una manera global, pudiendo enviar actualizaciones a través de configuraciones, al igual que mandar la realización de diferentes acciones.


