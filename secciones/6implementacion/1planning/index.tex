Una buena realización de un proyecto comienza con un buen diseño inicial en el que se establecen las bases para un desarrollo correcto.

Este proyecto puede ser dividido en 3 subproyectos, donde cada subproyecto tendrá una arquitectura diferente debido a las necesidades que requiere su implementación.

El primer subproyecto es el asignado a la implementación de un software que será desplegado en el dispositivo y que permita la conexión con el servidor, aceptando la manipulación del propio dispositivo de manera remota.

El segundo subproyecto corresponderá con el propio sistema alojado en el servidor, al que llamaremos backend, el cual tendrá acceso a una base de datos y servirá una API REST para permitir una comunicación entre los dispositivos y los administradores.

El tercer subproyecto corresponde con el desarrollo de un sitio web, al que se llamará frontend, el cual permite el acceso a los administradores para el control y gestión de los dispositivos, al igual que para consultar las estadísticas.