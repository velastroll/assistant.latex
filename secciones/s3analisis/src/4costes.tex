Para poder dar un valor monetario al proyecto, hay que tener en cuenta sus costes tanto económicos como de desarrollo para poder ser realizado.

\begin{enumerate}
    \item \textbf{Duración estimada del proyecto} \label{duracion_proyecto}
    
El proyecto descrito en el presente documento es un proyecto relativo a un Trabajo de Fin de Grado (TFG), por lo tanto su estimación de horas de elaboración deberán corresponder con el coste de horas estimadas en cuanto al valor hora-credito de la carrera.

Según la guía docente de la Universidad de Valladolid, cada crédito tiene una estimación de de trabajo de 25 horas, y un TFG tiene asignado una cuantía de 12 créditos.

Por tanto, el trabajo estimado para la elaboración de un TFG es:

\begin{center}
    \textit{12 créditos * 25horas/crédito = 300 horas}
\end{center}

    \item\textbf{ Coste de personal:}
    
El salario medio para un desarrollador junior en España que todavía no tiene el título está en \textit{18966euros/año}, y calculando que al año se hacen unas 1900 horas de trabajo de media, este desarrollador junior debería cobrar:

\begin{center}
    \textit{18966euros/1año * 1año/1900horas = 10euros/hora}
\end{center}

Teniendo en cuenta que la elaboración del proyecto tiene una estimación de 300 horas, el coste de personal será de:

\begin{center}
    \textit{10 euros/hora * 300 horas = 3000 euros}
\end{center}

    \item \textbf{ Coste del equipo: }
    
Para el desarrollo del proyecto se necesita la adquisición de dos dispositivos completos.

El coste del kit que proporciona la plataforma de Snips Seeed es de 115 euros por cada dispositivo, por lo que es necesaria la inversión de:

\begin{center}
    \textit{115 euros/dispositivo * 2 dispositivos = 230 euros}
\end{center}

    \item \textbf{Coste de suministros}
    
    La duración del proyecto a desarrollar por el desarrollador, teniendo en cuenta que realice una jornada de 8 horas al día, será:
    
\begin{center}
    \textit{1900 horas/año / 12 meses/año = 160 horas/mes}
    \textit{300 horas/tfg / 160 horas/mes = 1,875 meses/TFG}
\end{center}

    Como el gasto en suministros son cuotas mensuales, se aproxima a 2 meses de trabajo completo la resolución del TFG.
    
    El coste medio de consumo de suministros contando el acceso a internet y el consumo de la luz suma una cuantía de 25 euros al mes. Por tanto, el coste de suministros, será:
\begin{center}
    \textit{2 meses/TFG * 25 euros/mes = 50 euros/TFG}
\end{center}
    
\end{enumerate}

Teniendo en cuenta, según el libro de SPM ******, un proyecto tiende a retrasarse entre un 8\% y un 10\%, por lo que el coste total deberá ser recalculado, siendo el coste total la suma de todos los costes:

\begin{center}
    \textit{3000 + 230 + 50 = 3280 euros}
    \textit{3280 euros * 10\%  = 3608 euros}
\end{center}

El coste para la elaboración del proyecto, será por tanto la cuantía de \textbf{3608 euros}.