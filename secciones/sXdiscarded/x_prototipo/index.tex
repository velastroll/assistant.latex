% REQUISITOS
\subsection{Requisitos}

En relación a este proyecto, no nos adentraremos en cómo funciona el asistente, ni qué configuración del asistente vamos a instalar para que responda a eventos, si no de cómo va a ser controlado remotamente, y cómo va a ser desarrollado de manera que se puedan extender las funciones a realizar sobre el dispositivo en un futuro, tales como la instalación de actualizaciones, cambios de configuración o reinicios.

    \subsubsection{Conexión a Internet}

Como se ha nombrado previamente***, se busca la posibilidad de una conexión mínima a la red. Esta búsqueda de una conexión mínima es provocada por la posible utilización de la red de una tarjeta SIM, lo que supondría un sobrecoste mensual que podría dispararse.

Evitar una conexión a Internet y permitir un control remoto nos plantea las siguientes cuestiones:

\begin{enumerate}
    \item ¿Conectamos el dispositivo a la red del hogar? 
    
    Aquí se abre la posibilidad de configurar cada dispositivo a la red de su respectivo hogar, lo que supone que sería un requisito necesario que esos hogares dispongan de una conexión a Internet. Si el dispositivo está orientado a personas de edad avanzada, ¿podemos asegurar que esas personas van a tener conexión en sus casas? Como la respuesta a esta pregunta es negativa, se descartaría esta propuesta.
    
    \item Si conectamos el dispositivo a la red, ¿cómo nos conectamos a él desde el servidor?
    
    Para poder establecer la conexión desde el servidor a cada dispositivo con el fin de poder monitorizarlo y controlarlo, sería necesario no solo conectarlo a la red, sino que se necesitaría poder configurar el propio router de cada hogar con la finalidad de conseguir abrir un puerto al cual conectarnos remotamente, lo que supone tanto un trabajo demasiado tedioso cuando el número de dispositivos sea grande, como de seguridad al dejar un puerto abierto permitiendo una conexión externa vulnerable.
    Estas razones han sido por las cuales esta opción ha sido descartada como la principal, de modo que se establece la opción de una conexión a través de la red de datos otorgada por una tarjeta SIM.  
    
    \item Si conectamos el dispositivo a la red de la SIM, ¿cómo evitamos un sobrecoste de datos?
    
    Para monitorizar el dispositivo solamente hay que saber si está conectado, y para ello, el dispositivo puede contactar con el servidor mandándole una simple petición al día que correspondería a un \textit{ I'm Alive!}, de modo que el servidor guardaría esas peticiónes registrando el estado de cada dispositivo.
    
    En caso de que el dispositivo tenga algún problema o detecte cualquier aviso, simplemente tendría que cambiar el tipo de petición que realice al servidor con otro mensaje diferente, como \textit{Hey, technical problem!}, pudiendo ser reflejado en el servidor y tomando las medidas oportunas.
    
    Este tipo de conexión tendría un uso de red mínimo, evitando los sobrecostes, pero nos genera la siguiente y última cuestion.
    
    \item Vale, el dispositivo está conectado, pero... ¿cómo lo controlamos?
    
    Aquí es donde ocurre la magia. La conexión al servidor estará montada bajo una arquitectura a cuyos servicios se accederan mediante una API REST***.
    
    El aviso al servidor del \textit{I'm Alive!} será entonces una llamada de tipo GET a un end-point de la API, y toda llamada GET puede devolver junto al estado de la llamada, un cuerpo de mensaje.
    Es en ese cuerpo de la respuesta donde se incluirá la configuración o evento necesario para el dispositivo, en caso de que haya alguno a la espera de ser realizado.
    
    Como en el cuerpo de la respuesta se podría introducir configuraciones, se podría incluso incluir la configuración de la red WiFi del hogar en el cual está instalado, liberándonos completamente de los cargos por uso de la red de la SIM, al igual que también se podría cambiar el tiempo que tarda el dispositivo en volver a avisar de su estado, permitiendo una comunicación entre servidor y dispositivo más rápida.
    
\end{enumerate}

    \subsubsection{Seguridad contra la suplantación de identidad}

    El servidor que albergue nuestro sistema estará preparado para recibir información de cada dispositivo, pero tiene que ser consecuente de que la información recibida es del dispositivo de dice ser, evitando posibles falseamientos de datos.
    
    Para evitar la suplantación de identidad, toda petición realizada por el dispositivo deberá ser autenticada a través de un token puesto en su cabecera, en el campo de \textit{Authorization}. El token introducido tiene una duración temporal, habiendo sido obtenido tras previamente haber iniciado sesión.
    
    Para el inicio de sesión, cada dispositivo utilizará su dirección MAC como nombre de usuario, siendo esta una dirección única para cada tarjeta de red, la cual ya viene integrada en la RaspberryPi3B.
    
    La contraseña será por tanto una encriptación de la dirección MAC de cada dispositivo, que será desencriptada y comprobada en el servidor, otorgando en la respuesta dos tokens: uno de acceso, y otro de refresco.

% DESARROLLO
\subsection{Casos de uso}

% DIAGRAMAS DE ESTADOS
\subsection{Diagramas de estados}