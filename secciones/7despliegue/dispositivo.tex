La instalación del software en el dispositivo podrá ser flasheando una imagen con el sistema operativo y el asistente ya cargado.

La guía por tanto, no entrará en ese apartado ino en la posibilidad de instalar el asistente en cualquier dispositivo que disponga de una arquitectura Unix.

Para ello, se introducirá el repositorio \textbf{assistant.task} en el dispositivo, dentro del directorio \text y aprovechando los scripts contenidos en él, se instalará todo lo necesario.

\begin{enumerate}
    \item Acceso al repositorio:
    
        \begin{lstlisting}[language=bash]
            $ cd /home/pi/assistant.task
        \end{lstlisting}
    
    \item Instalacción de dependencias
    
        \begin{lstlisting}[language=bash]
            $ sh requirements.sh
        \end{lstlisting}
    
\end{enumerate}

Tras la ejecucción del script ya se instalan las dependencias requeridas y se configura el inicio automático del controlador remoto del asistante en cada reinicio de la máquina.
