También conocido como Alexa, es la opción creada por Amazon que más fuerza está tomando en la sociedad para ser elegida como la asistente de voz inteligente que nos acompañe en nuestro día a día.

A pesar de todas las ventajas posibles similares a las que se han descrito para el asistente de Google, también comparte sus contras, como las referentes a la localización donde se procesan los audios y donde se efectúa el entrenamiento, por lo que es otra opción que va a ser rechazada.

Sin tener en cuenta la ubicación del proceso de penamiento, Amazon asegura que almacena todos los audios hasta que es el propio usuario quien decide borrarlos, pero aún con la solicitud expresa del usuario, no pueden ser borrados completamente ya que han sido compartidos con terceros, que son quienes han desarrollado funciones específicas, llamadas Skills, y que son ellos quien pueden tener los datos del usuario almacenados, sin que el propio Amazon sea capaz de borrarlos aunque quisiese.

https://www.elmundo.es/tecnologia/2019/07/04/5d1ccf42fc6c833f3f8b460d.html