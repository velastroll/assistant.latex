El asistente inteligente de Google es el que está en el año 2020 a la cabeza de los asistentes virtuales********ENLACE********, y esto es debido a que viene instalado en todos los dispositivos Android, ocupando estos dispositivos la mayor parte del mercado de la telefonía móvil. Esta posición privilegiada le da un mayor entrenamiento, elevando el acierto y usabilidad de este.

La creación de Google tiene una gran aceptación con los diferentes aparáticos de la domótica del hogar, facilitando la implementación y realización de tareas:

- Una tarea es un conjunto de acciones que se llevan a cabo tras accionarse un evento que hace de interruptor, pudiendo ser el evento tanto un comando de voz, la pulsación de un interruptor o la activación de un sensor, entre otras opciones.

De esta manera, Google permite el control de la domótica del hogar, o la realización de diferentes acciones en nuestro día a día, pero requiere ser configurado por cada usuario, evitando por tanto la posibilidad de un despliegue común.

\subsubsection{Cómo funciona}

El proceso de pensamiento que hace el asistente de Google está basado en la nube, pero no solo eso sino que Google hace en sus servidores también el proceso de Speech-to-Text.

De este modo, Google almacena todos los audios que se le mandan a través de su asistente, estudiándolos uno a uno y asegurando o corrigiendo sobre la respuesta que envió el asistente.

Este proceso de corregir o confirmar es el método que tiene Google de entrenamiento, de manera que en la próxima consulta sobre el mismo tema, el asistente pueda responder con mayor exactitud.

Queda claro que sobre la teoría es un buen plan de entrenamiento, y en un mundo ideal esto sería perfecto, pero esos audios también pueden contener parte de información privada, o pueden servir para espiar conversaciones privadas que un usuario puede no querer que estén almacenadas, ni que sean escuchadas por otra persona, aunque sea un propio trabajador.

https://elpais.com/tecnologia/2019/07/18/actualidad/1563466196_049167.html

En cuanto a la posibilidad de ser desplegado en otro dispositivo que no sea un smartphone o una tablet, Google nos lo pone fácil, ya que otorga una librería con la cuál facilita la instalación y uso en una gran variedad de dispositivos, cuyo requisito es que dispongan de conexión a Internet.

