Ante la falta de privacidad por parte de Google y Amazon, se abre la puerta a Mycroft, proyecto open source que busca como pilar la seguridad de los datos y la protección de la privacidad, de manera que asegura no almacenar ningún dato o información del usuario potencial.

Mycroft tiene la mayor comunidad para el desarrollo e implementación de asistentes, de modo que dispone de una gran cantidad de skills que añadir a nuestro dispositivo, posicionándose como opción principal para la elaboración del proyecto.

Este asistente virtual puede ser fácilmente implementado en dispositivos tales como Raspberry, cumpliendo los propósitos y requisitos de este proyecto.

En cuanto a la conexión a Internet, el equipo de Mycroft informa acerca de estar trabajando en una herramienta que permita desplegar el asistente ya entrenado dentro del dispositivo, evitando la conexión, pero de momento esa herramienta no está finalizada, por lo que cojea en ese aspecto y de momento, debería tener una conexión permanente.

En caso de que estén en lo cierto y estén trabajando en esa herramienta, este asistente ocuparía la primera opción como asistente a desplegar, pero en el momento actual en el cual este proyecto cobra vida, no es viable su implementación.

