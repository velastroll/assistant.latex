Una vez que ya se ha expuesto cual es el funcionamiento base de un asistente virtual inteligente, interesa saber cuál es de todos los existentes que más se adecúa a las necesidades establecidas para este proyecto, de manera que se le exigirá que cumpla el máximo de los siguientes puntos expuestos.

\subsubsection{Despliegue en Dispositivos}

El asistente seleccionado debe poder ser desplegado de manera gratuita en un dispositivo formado por una placa RasbperryPi, o similar.

Para ello, se requiere que el asistente disponga de una librería o de un modo de uso que se pueda implementar en un dispositivo pequeño y portátil, facilitando el cambio de una ubicación física por otra de los distintos posibles puntos que existen dentro de un hogar, manteniendo una instalación lo más simple y limpia posible, en la que solo sea requerida la localización de un enchufe activo que posea corriente eléctrica.

\subsubsection{Tratamiento de la Información}

El asistente debe seguir la Ley Orgánica de Protección de Datos~\cite{lopd} y el Reglamento Europeo de Protección de Datos~\cite{rgpd}, y no debe tener acceso a escuchar conversaciones ajenas violando la intimidad de los usuarios.

Cualquier vacío legal o conjunto de cláusulas de extensa longitud no podrá será aceptado, para poder asegurar la protección de los datos de los usuarios.

En caso de almacenar algún tipo de dato, debe poder ser público para el usuario en cuestión, o haber sido aceptado expresamente por el usuario a favor de un control de su seguridad.

\subsubsection{Conexión a Internet}

La orientación de un dispositivo asistente inteligente como este a personas de una edad avanzada debe tener en cuenta que es posible que la mayoría de sus usuarios potenciales no tengan una conexión a internet en sus respectivos hogares.

Esto provoca que la conexión del dispositivo a la red tiene que ser lo mínima posible, favoreciendo los asistentes en los cuales el proceso de pensamiento sea ejecutado dentro del propio dispositivo, evitando tener que hacer el cálculo e identificación en servidores de alojados en la nube.

Es decir, el dispositivo debe ser capaz de funcionar en el lugar más remoto posible y sin ninguna conexión a internet, simplemente tras ser conectado a una red eléctrica.

\subsubsection{Adicción de Nuevas Tareas}

El software del asistente inteligente debe tener la capacidad de añadir nuevas tareas fácilmente, de modo que se puedan añadir nuevas funciones tanto propias, como desarrolladas por la comunidad de Internet.