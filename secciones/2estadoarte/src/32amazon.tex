~También conocido como Alexa, es la opción creada por Amazon que más fuerza está tomando en la sociedad para ser elegida como el asistente de voz inteligente que nos acompañe en nuestro día a día.

A pesar de todas las ventajas posibles similares a las que se han descrito para el asistente de Google, también comparte sus contras, como las referentes a la localización donde se procesan los audios y donde se efectúa el entrenamiento, por lo que es otra opción que va a ser rechazada.

Sin tener en cuenta la ubicación del proceso de pensamiento, Amazon asegura\cite{escandalo-amazon} que almacena todos los audios hasta que es el propio usuario quien decide borrarlos, pero aún con la solicitud expresa del usuario no pueden ser borrados completamente ya que han sido compartidos con terceros, que son quienes han desarrollado funciones específicas, llamadas Skills. Estos desarrolladores no solo tienen acceso a los audios sino que los poseen, pudiendo hacer una mala práctica con ellos.

Por tanto, aunque el usuario en cuestión pida a Amazon el borrado de estos ficheros, la compañía únicamente podría borrar los archivos que posee, permaneciendo todos los cuales un desarrollador de estas skills haya almacenado, sin permitir que el propio Amazon sea capaz de eliminarlos aunque tratase de hacerlo.