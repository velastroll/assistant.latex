\subsection{Diagrama de casos de uso}

\subsection{Definición de casos de uso}

%% ====== DISPOSITIVO ======

%% IDENTIFICAR DISPOSITIVO
\begin{table}[]
\centering
	\setlength{\extrarowheight}{3pt}
		\begin{tabular}{rc{1.85cm}|c{8cm}}
	    \hline
        \multicolumn{1}{|l}{ \textbf{CU-D-01}}     & \multicolumn{1}{c|}{\textbf{Inicio de sesión de dispositivo}  } \\
	    \hline \hline
	    \multicolumn{1}{|l}{Descripción}     & \multicolumn{1}{Y|}{
	    Un dispositivo se identifica ante el sistema con el fin de obtener unos tokens de acceso, con los que autenticar posteriormente sus llamadas.
	    }  \\ \cline{1-2}
	    \multicolumn{1}{|l}{Actor}           & \multicolumn{1}{l|}{Dispositivo}  \\ \cline{1-2}
	    \multicolumn{1}{|l}{Precondiciones}  & \multicolumn{1}{l|}{El dispositivo tiene conexión a Internet.}  \\ \cline{1-2}
	    \multicolumn{1}{|l}{Postcondiciones} & \multicolumn{1}{l|}{Se obtiene un par de tokens.}\\ \cline{1-2}
	    \multicolumn{1}{|l}{Flujo normal}    & \multicolumn{1}{Y|}{
1. El dispositivo obtiene su dirección MAC.
2. El dispositivo genera la clave a partir de una encriptación de su MAC, formando sus credenciales, y se las envía al servidor. \newline
3. El servidor comprueba la validez de las credenciales, genera un par de tokens asociados al dispositivo y los devuelve. \newline
	    } \\ \cline{1-2}
	    \multicolumn{1}{|l}{Flujo Alternativo} & \multicolumn{1}{l|}{}\\ \cline{1-2}
	    \hline
	    \end{tabular}
	\label{table:cu-d-01}
	\caption{Inicio de sesión de dispositivo}
\end{table}

%%%% REFRESH TOKEN
\begin{table}[]
\centering
	\setlength{\extrarowheight}{3pt}
		\begin{tabular}{rc{1.85cm}|c{8cm}}
	    \hline
        \multicolumn{1}{|l}{ \textbf{CU-D-02}}     & \multicolumn{1}{c|}{\textbf{Refrescar token de dispositivo}  } \\
	    \hline \hline
	    \multicolumn{1}{|l}{Descripción}     & \multicolumn{1}{Y|}{
	    Un dispositivo actualiza su par de tokens a partir de un token de refresco.
	    }  \\ \cline{1-2}
	    \multicolumn{1}{|l}{Actor}           & \multicolumn{1}{l|}{Dispositivo}  \\ \cline{1-2}
	    \multicolumn{1}{|l}{Precondiciones}  & \multicolumn{1}{Y|}{
1. El dispositivo tiene conexión a Internet.\newline
2. El dispositivo tiene un token de refresco válido.}  \\ \cline{1-2}
	    \multicolumn{1}{|l}{Postcondiciones} & \multicolumn{1}{l|}{Se obtiene un par de tokens.}\\ \cline{1-2}
	    \multicolumn{1}{|l}{Flujo normal}    & \multicolumn{1}{Y|}{
1. El dispositivo manda su token de refresco. \newline
2. El sistema comprueba la validez del token de refresco, genera un par de tokens nuevos y los devuelve. \newline
3. El dispositivo actualiza los antiguos tokens por los nuevos.
	    } \\ \cline{1-2}
	    \multicolumn{1}{|l}{Flujo Alternativo} & \multicolumn{1}{l|}{Diagramas/ Fotos}\\ \cline{1-2}
	    \hline
	    \end{tabular}
	\label{table:cu-d-02}
	\caption{Refrescar token}
\end{table}

%%%% AVISO ACTIVO
\begin{table}[]
\centering
	\setlength{\extrarowheight}{3pt}
		\begin{tabular}{rc{1.85cm}|c{8cm}}
	    \hline
        \multicolumn{1}{|l}{ \textbf{CU-D-04}}     & \multicolumn{1}{c|}{\textbf{Aviso de dispositivo encendido} } \\
	    \hline \hline
	    \multicolumn{1}{|l}{Descripción}     & \multicolumn{1}{Y|}{
	    Un dispositivo avisa de que está en funcionamiento.
	    }  \\ \cline{1-2}
	    \multicolumn{1}{|l}{Actor}           & \multicolumn{1}{l|}{Dispositivo}  \\ \cline{1-2}
	    \multicolumn{1}{|l}{Precondiciones}  & \multicolumn{1}{Y|}{
1. El dispositivo tiene conexión a Internet.\newline
2. El dispositivo tiene un par de tokens activos.}  \\ \cline{1-2}
	    \multicolumn{1}{|l}{Postcondiciones} & \multicolumn{1}{l|}{}\\ \cline{1-2}
	    \multicolumn{1}{|l}{Flujo normal}    & \multicolumn{1}{Y|}{
1. El dispositivo hace una petición \textit{/alive} al servidor. \newline
2. El sistema realiza el caso de uso \textbf{*****}. \newline
3. El dispositivo recibe una respuesta vacía del servidor. \newline
4. Finaliza el caso de uso.
	    } \\ \cline{1-2}
	    \multicolumn{1}{|l}{Flujo Alternativo} & \multicolumn{1}{l|}{
4.1. El dispositivo recibe en la respuesta una tarea pendiente. \newline
4.2. El dispositivo avisa al servidor que va a hacer la tarea. \newline
4.3. El dispositivo hace la tarea. \newline
4.4. Finaliza el caso de uso.
	    }\\ \cline{1-2}
	    \hline
	    \end{tabular}
	\label{table:cu-d-03}
	\caption{Dispositivo notifica que está encendido.}
\end{table}

%%% ====== SISTEMA =======

%%%% COMPROBAR CREDENCIALES

\begin{table}[]
\centering
	\setlength{\extrarowheight}{3pt}
		\begin{tabular}{rc{1.85cm}|c{8cm}}
	    \hline
        \multicolumn{1}{|l}{ \textbf{CU-S-01}}     & \multicolumn{1}{c|}{\textbf{Identificación de dispositivo} } \\
	    \hline \hline
	    \multicolumn{1}{|l}{Descripción}     & \multicolumn{1}{Y|}{
	    El sistema debe identificar a un dispositivo ante el sistema, y en caso de no estar identificado, registrarle.
	    }  \\ \cline{1-2}
	    \multicolumn{1}{|l}{Actor}           & \multicolumn{1}{l|}{Servidor}  \\ \cline{1-2}
	    \multicolumn{1}{|l}{Precondiciones}  & \multicolumn{1}{Y|}{
1. El servidor recibe una llamada para identificar a un dispositivo.}  \\ \cline{1-2}
	    \multicolumn{1}{|l}{Postcondiciones} & \multicolumn{1}{l|}{
1. Se genera un par de tokens asociados a un dispositivo concreto.\newline
2. Existe un registro del dispositivo en el sistema.}\\ \cline{1-2}
	    \multicolumn{1}{|l}{Flujo normal}    & \multicolumn{1}{Y|}{
1. El servidor recibe unas credenciales en una llamada. \newline
2. El servidor comprueba la validez de los credenciales. \newline
3. El dispositivo crea un par de tokens de acceso de tipo dispositivo y los devuelve por la llamada. \newline
4. Finaliza el caso de uso.
	    } \\ \cline{1-2}
	    \multicolumn{1}{|l}{Flujo Alternativo} & \multicolumn{1}{l|}{
2.1. El servidor detecta que el dispositivo aún no está registrado y lo registra.
	    }\\ \cline{1-2}
	    \hline
	    \end{tabular}
	\label{table:cu-d-03}
	\caption{Servidor identifica a un dispositivo.}
\end{table}

\subsection{Matriz de aproximación con requisitos}