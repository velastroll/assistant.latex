Para el desarrollo de este proyecto se ha escogido una implementación siguiendo la combinación de un modelo iterativo y un modelo incremental.~\cite{mod-it-inc}

Este método puede describirse de una manera simple y rápida, basada en la repeticion de ciclos en los cuales se van mejorando las funcionalidades implementadas a partir de la experiencia del usuario, y añadiendo nuevas.

Cada ciclo puede describirse como la continuación de los siguientes pasos:

\begin{enumerate}
    \item El desarrollador muestra un prototipo.
    \item El usuario lo prueba.
    \item El usuario muestra su experiencia tanto positiva como negativa, comentando posibles mejoras y proponiendo nuevas funcionalidades.
    \item Se subsanan esas carencias.
    \item Se obtiene una nueva versión del prototipo.
    \item Vuelta al paso inicial.
\end{enumerate}{}

La elección de este modelo se debe, por tanto, a la naturaleza del proyecto del cual se trata: al estar trabajando con unas nuevas tecnologías, y no tener muy claras las capacidades de las cuales los posibles asistentes están dotados, es preferible realizar un desarrollo basado en la entrega de pequeños prototipos que nos vayan ampliando el conocimiento de todas las posibilidades que van a poder ser implementadas, al igual que produce una retroalimentación a partir de la experiencia del usuario, siendo bastante útil tanto para mejorar, como para eliminar o para añadir nuevas funcionalidades.

En el caso del presente proyecto, el usuario el cual realiza las pruebas del prototipo se trata del propio tutor del proyecto, el cual otorga su experiencia, experiencia que está curtida a través de años de trabajo y perfeccionamiento, siendo más sabia que la del propio desarrollador, ayudando a elaborar un rediseño de las características que aumente la usabilidad. 