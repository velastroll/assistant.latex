\documentclass[openright,twoside,10pt]{book}
\usepackage[b5paper,left=2cm,top=2.5cm,right=1.5cm,bottom=2.5cm]{geometry} 
\usepackage[spanish]{babel} % espanol
\usepackage[utf8]{inputenc} % acentos sin codigo
\usepackage{graphicx} % gráficos
\usepackage{lscape}
\usepackage{fancyvrb}
\usepackage{fancyhdr}
\usepackage{wrapfig}
\usepackage{multirow}
\usepackage{rotating} % rotate figures
\usepackage{array} % for defining a new column type
\usepackage{varwidth} % for the varwidth minipage environment
\newcolumntype{S}{>{\begin{varwidth}{2cm}}c<{\end{varwidth}}}
\newcolumntype{M}{>{\begin{varwidth}{4cm}}c<{\end{varwidth}}}
\newcolumntype{L}{>{\begin{varwidth}{6cm}}c<{\end{varwidth}}}
\newcolumntype{X}{>{\begin{varwidth}{8cm}}c<{\end{varwidth}}}
\newcolumntype{Y}{>{\begin{varwidth}{10cm}}l<{\end{varwidth}}}
\usepackage{float}
\usepackage{listings} %% START: Definimos el modo de mostrar codigo
\usepackage{color}
%\usepackage{bera}% optional: just to have a nice mono-spaced font
\usepackage{xcolor}
\usepackage{enumitem}
\usepackage[table,xcdraw]{xcolor}
\colorlet{punct}{red!60!black}
\definecolor{background}{HTML}{EEEEEE}
\definecolor{delim}{RGB}{20,105,176}
\colorlet{numb}{magenta!60!black}
\definecolor{dkgreen}{rgb}{0,0.6,0}
\definecolor{gray}{rgb}{0.5,0.5,0.5}
\definecolor{mauve}{rgb}{0.58,0,0.82}
\lstset{frame=tb, % VIM
  language=java,
  aboveskip=3mm,
  belowskip=3mm,
  showstringspaces=false,
  columns=flexible,
  basicstyle={\small\ttfamily},
  numbers=none,
  numberstyle=\tiny\color{gray},
  keywordstyle=\color{blue},
  commentstyle=\color{dkgreen},
  stringstyle=\color{mauve},
  breaklines=true,
  breakatwhitespace=true,
  tabsize=2
}
\lstdefinelanguage{json}{ % Define JSON
    basicstyle=\normalfont\ttfamily,
    numbers=left,
    numberstyle=\scriptsize,
    stepnumber=1,
    numbersep=8pt,
    showstringspaces=false,
    breaklines=true,
    frame=lines,
    backgroundcolor=\color{background},
    literate=
     *{0}{{{\color{numb}0}}}{1}
      {1}{{{\color{numb}1}}}{1}
      {2}{{{\color{numb}2}}}{1}
      {3}{{{\color{numb}3}}}{1}
      {4}{{{\color{numb}4}}}{1}
      {5}{{{\color{numb}5}}}{1}
      {6}{{{\color{numb}6}}}{1}
      {7}{{{\color{numb}7}}}{1}
      {8}{{{\color{numb}8}}}{1}
      {9}{{{\color{numb}9}}}{1}
      {:}{{{\color{punct}{:}}}}{1}
      {,}{{{\color{punct}{,}}}}{1}
      {\{}{{{\color{delim}{\{}}}}{1}
      {\}}{{{\color{delim}{\}}}}}{1}
      {[}{{{\color{delim}{[}}}}{1}
      {]}{{{\color{delim}{]}}}}{1},
}

\usepackage[table,xcdraw]{xcolor} % tabla con colores
\setlength{\parskip}{10pt plus 1pt minus 1pt}
 % aqui definimos el encabezado de las paginas pares e impares.
\rhead[]{}

\renewcommand{\headrulewidth}{0.5pt}

% aqui definimos el pie de pagina de las paginas pares e impares.
\rfoot[\thepage]{\thepage}
\cfoot[]{}
\renewcommand{\footrulewidth}{0pt}

%redefino el verbatim
%\renewenvironment{verbatim}{\begin{Verbatim}[frame=single,fontsize=\small]}{\end{Verbatim}}

% aqui definimos el encabezado y pie de pagina de la pagina inicial de un capitulo.
\fancypagestyle{plain}{
\fancyhead[R]{}
\fancyfoot[C]{}
\fancyfoot[R]{\thepage}
\renewcommand{\headrulewidth}{0.5pt}
\renewcommand{\footrulewidth}{0pt}
}

\pagestyle{fancy} % seleccionamos un estilo

\date{21 de Enero de 2020}
\author{Velasco Gil, Álvaro}
\title{TFG: titulo}

\begin{document}

\begin{titlepage}

\begin{center}
\vspace*{-1in}
\begin{figure}[htb]
\begin{center}
\includegraphics[width=3cm]{./img/logo}
\end{center}
\end{figure}
\begin{large}
\textbf{Universidad de Valladolid}
\end{large}

\vspace*{0.15in}

\vspace*{0.6in}
\begin{large}
\textbf{ESCUELA DE INGENIERÍA INFORMÁTICA}

\end{large}
\vspace*{0.2in}
\textbf{ GRADO EN INGENIERÍA INFORMÁTICA}\\
\textbf{ MENCIÓN EN INGENIERÍA DEL SOFTWARE }
\vspace*{0.1in}
\rule{140mm}{0.1mm}\\
\vspace*{0.2in}
\begin{large}
\textbf{{\LARGE DISEÑO E IMPLEMENTACIÓN DE UN SISTEMA DE CONTROL Y MONITORIZACIÓN DE ASISTENTES INTELIGENTES\\}}
\end{large}
\vspace*{0.2in}
\rule{140mm}{0.1mm}\\
\vspace*{2in}
\begin{large}
\begin{flushright}
\textbf{Alumno: Velasco Gil, Álvaro \\
\vspace*{0.3in}
Tutor: Vegas, Jesús }
\end{flushright}
\end{large}
\end{center}

\end{titlepage}

\newpage
\mbox{}	
\thispagestyle{empty} % para que no se numere esta página

\chapter*{}
\pagenumbering{Roman} % para comenzar la numeración de paginas en números romanos
\begin{flushright}
\textit{%Dedicatoria,\\
Dedicado a mis padres y mi hermana, por confiar en mi cuando yo no lo hacía.}
\end{flushright}


\chapter*{Agradecimientos} % si no queremos que añada la palabra "Capitulo"
\addcontentsline{toc}{chapter}{Agradecimientos} % si queremos que aparezca en el índice
\markboth{AGRADECIMIENTOS}{AGRADECIMIENTOS} % encabezado 
Quiero agradecer en primer lugar a mi tutor, por la paciencia,

a mis amigo Jorge Sanz por facilitarme los primeros cursos,

a Javier Helguera por hacerme la carrera más llevadera,

a Diego Dominguez y David Escarda por hacer más sencillas las noches estudiando en el aulario,

y a mi familia, por soportar mis cambios de humor.
% Aquí agradecer

\chapter*{Resumen} % si no queremos que añada la palabra "Capitulo"
\addcontentsline{toc}{chapter}{Resumen} % si queremos que aparezca en el índice
\markboth{RESUMEN}{RESUMEN} % encabezado
\begin{flushleft}

El fin de este proyecto es el de crear un sistema capaz de monitorizar y controlar remotamente diferentes dispositivos que serán utilizados como asistentes inteligentes.

Para el desarrollo de este proyecto se ha utilizado diferentes lenguajes, como Kotlin en el backend, Python en los dispositivos, y VueJS, HTML y CSS en el frontend, siguiendo un plan basado en iteraciones.

% Aquí el resumen

\end{flushleft}

\chapter*{Abstract} % si no queremos que añada la palabra "Capitulo"
\addcontentsline{toc}{chapter}{Abstract} % si queremos que aparezca en el índice
\markboth{ABSTRACT}{ABSTRACT} % encabezado
\begin{flushleft}

% Aquí va el abstract

\end{flushleft}

\tableofcontents % indice de contenidos

\cleardoublepage
\addcontentsline{toc}{chapter}{Lista de Figuras} % para que aparezca en el indice de contenidos
\listoffigures % indice de figuras

\cleardoublepage
\addcontentsline{toc}{chapter}{Lista de Tablas} % para que aparezca en el indice de contenidos
\listoftables % indice de tablas

\chapter{Introducción}\label{cap.introduccion}
\pagenumbering{arabic} % para empezar la numeración con números
\section{Introducción}

En este proyecto se abordará el proceso de control y monitorización remota de dispositivos asistentes inteligentes.

Tras la lectura de este documento se comprenderán tanto los motivos por los cuales se ha decidido tomar esta opción, como el proceso de despliegue del sistema, pasando por las fases de implementación en las que se enseñará a replicar proyectos de estructura similar, como por las fases de desarrollo en la cuales se estudian los posibles riesgos y características del proyecto, al igual que por el plan de desarrollo en el cual se programa toda la elaboración.

\section{Motivación}

La idea de este proyecto llegó impulsada gracias a dos grandes motivaciones: conseguir la privacidad en la utilización de dispositivos inteligentes, y la posibilidad de asistir a la población envejecida de entornos rurales.

La primera motivación, relativa a la privacidad viene dada ya que los asistentes inteligentes están en auge y las grandes empresas están dando acceso a estos servicios de manera gratuita, donde lo único que hay que hacer para disfrutar de uno de ellos es pagar es el dispositivo físico, asumiendo únicamente los costes del hardware a un precio reducido, lo que hace cuestionar la idea del modelo de negocio que están siguiendo para que salga rentable: El desarrollo de un sistema software capaz de dar respuesta a toda cuestión que un usuario se llegue a plantear, al igual que el mantenimiento de las infraestructuras que puedan soportar todo un sistema capaz de retroalimentarse mediante la información vertida por los usuarios, tiene un coste muy elevado como para ofrecer ese servicio de manera gratuita. Un modelo de negocio en el cual  se ofrece un dispositivo a un bajo precio con unas prestaciones de gran nivel sin cobrar a mayores una cuota mensual da que pensar, por ejemplo, que el producto mediante el cual está basado su modelo de negocio no es el que el usuario cree: Para el usuario, el producto es el dispositivo que le va a servir un uso, pero para la empresa, el producto es toda la información personal que le va a proporcionar cada usuario que utilice el dispositivo. La información recopilada a través de los asistentes, al ser obtenida directamente de los hogares de cada usuario no es solamente privada, si no que también es íntima, lo que hace cuestionarse si realmente la comunidad de usuarios sabe que hay empresas aprovechándose de las conversaciones de su día a día para recolectar toda la información de un usuario que fluye en los lugares más personales de cada hogar, capturando información con gran potencial, ya sea sobre gustos, tendencias, necesidades o inclinaciones políticas, con el fin de poder no solo utilizar esos datos, sino también pudiendo incluso venderlos a terceros, a través de los cuales pueden llegar a intentar manipularnos, como ya se ha visto en casos como el escándalo de Cambridge Analytica.~\cite{cambridge} 

Esta aparente ignorancia de la población sobre el tratamiento de datos que las empresas podrían ser capaces de hacer genera la motivación suficiente para la creación de un nuevo dispositivo asistente, el cual haga más fácil la vida de un sesgo de la sociedad al cual iría orientado, sin tener la necesidad ni posibilidad de almacenar datos de carácter sensible.

Por otro lado también se observa otra gran motivación para la elaboración del proyecto, y es producida por la migración por la parte joven de la sociedad en los tiempos que acontecen actualmente, lo cual está provocando una despoblación en sus lugares de origen, dejando a los familiares de mayor edad en soledad, siendo la parte de la familia que a grosso modo necesita más atención y requiere más ayuda para pasar el día.

La implementación de un asistente inteligente el cual ayude a este conjunto de la sociedad a entretenerse proponiendo tanto eventos cercanos a ellos, como respondiendo sus dudas de una manera rápida, o sirviéndoles para buscar ayuda en caso de posible caída o solicitud de auxilio, puede ser una gran herramienta que mejore su calidad de vida.


\section{Objetivos}

Dadas y mencionadas las motivaciones que impulsan este proyecto, se puede destacar que el objetivo principal se basa en la privacidad por diseño.

El objetivo de este proyecto se logrará, por tanto, mediante el despliegue de un sistema que permita la conexión de dispositivos al servidor de manera automática y remota, a la vez que permita que sean controlados desde un sitio web de administración de una manera simple, desde la cual se podrá ver también estadísticas de uso para poder comprobar el estado de los usuarios, sin la obtención de datos que atenten contra su privacidad.

Estos dispositivos, como objetivo, estarán orientados a personas del rango de la tercera edad, estando cada dispositivo asignado a una persona, de la cual se conoce su localidad con el fin de obtener información relacionada a la hora de responder.

Los dispositivos podrán entonces ser controlados remotamente por parte de los administradores y podrán ser configurados tanto de manera individual, como en función de su localidad, o de una manera global, pudiendo solicitar al dispositivo la actualización de manera remota a través de configuraciones, o solicitar la realización de diferentes acciones como puede ser la auto-configuración de la red Wi-Fi, u otras para mejorar la gestión de despliegue sin requerir la conexión física con un ordenador.




\chapter{Estado del Arte}\label{cap.arte}
\section{Introducción}
Los asistentes inteligentes están buscando su hueco en todo hogar, y esto es un hecho.
Estos asistentes nos hacen la vida más sencilla, ayudando a actualizar y controlar toda la domótica de nuestras casas con acciones que hace unos años solo podíamos imaginar en películas de ciencia ficción.

La utilización de asistentes virtuales para facilitar acciones que se repiten de manera continuada a lo largo de los días está cada vez más de moda: según un estudio de la reputada Nielsen~\cite{nielsen}, durante el Q2 de 2018 ha llegado a crecer el número de dispositivos desplegados en los hogares de Estados Unidos hasta alcanzar el 24\% de la población.

Si nos preguntamos para qué puede querer el 24\% de las familias un dispositivo asistente en sus casas, se puede observar tras una encuesta que en la mayoría de hogares se daba un uso bastante simple, solicitando al dispositivo únicamente la reproducción de contenido multimedia, pero esto ocurre únicamente en las primeras semanas de uso. Posteriormente se tiende a dar más uso al dispositivo llegándole a requerir información habitual que se requiere de manera repetitiva cada semana, como puede ser la consulta del tiempo, la del tráfico antes de ir a trabajar o a la consulta de algún evento cercano importante.

La simplicidad de uso y todas las posibilidades que abarca ha hecho que una vez ha sido usado este tipo de dispositivo inteligente en cualquier hogar, se le siga utilizando cada vez para más tareas, dada su usabilidad y abanico de utilidades.

En el estudio se puede observar el comportamiento de los usuarios: una vez el dipositivo está en entre las cuatro paredes de la casa, se tiende a acompañar a este dispositivo inteligente de otros dispositivos domóticos, convirtiendo poco a poco el hogar en un sistema interconectado: cuatro de cada diez hogares disponen de más de un dispositivo inteligente en el hogar, y el 45\% de los hogares se plantean acompañar al que ya tienen, comprando otro.

Esta gran aceptación a la domótica viene dada gracias a la cantidad de variantes de las cuales podemos dotar al hogar, pudiendo ser todas controladas a través del dispositivo asistente, así como puede ser la programación de los tiempos de uso de aparatos como las bombillas, el termostato, los sistemas de seguridad, e incluso tanto el frigorífico, como bloquear las puertas y ventanas de la casa, al igual que mantener el registro de cuáles y cuándo han sido abiertas.

Al irrumpir estos dispositivos en las acciones cotidianas de un día corriente, se le permite al usuario adaptarse y ver que estos aparatos te acaban facilitando el día.

Viendo el creciente uso de estos dispositivos y las grandes marcas que hacen acto de presencia facilitando su implementación, entre las cuales se encuentran Google o Amazon, se puede pensar que este mercado ya está dominado y pertenece a estos dos grandes imperios, pero en este caso no hay que tener en cuenta todo lo que ofrecen, sino que hay que tener en cuenta todo lo que no están ofreciendo, o qué es lo que sus dispositivos no están preparados para ofrecer:

Como bien se ha visto, el punto de mira en todo el tema de la domótica y asistentes tiene un público objetivo que es el que pertenece a un rango de edad al que le gusta la tecnología y le parecería interesante automatizar las tareas del hogar, pero el verdadero público que podría exprimir todo su potencial es el público que menos acostumbrado nos tiene a estar a la vanguardia de la tecnología, el público perteneciente a la tercera edad.

El rango de personas de tercera edad es un público que puede necesitar que le levanten las persianas de manera automática porque quizá, le cuesta levantarse de la cama. Es un público que puede necesitar utilizar un asistente que le recuerde realizar ciertas tareas en su día a día como puede ser ir al médico, tomarse la pastilla, hacer la compra, o recordarle qué productos tiene que comprar, al igual que el cumpleaños de un ser querido. Todo esto acompañado de la seguridad que le puede proporcionar tareas como asegurarse que todas las puertas de su casa están bien cerradas, o saber que aún estando estas personas solas en casa, podrían solicitar auxilio en un momento de urgencia a traves del asistente.

La tecnología vinculada a Internet nos hace la vida más sencilla a través del smartphone o de los ordenadores, pero las labores que nos proporciona como extra un dispositivo asistente inteligente a un usuario medio el cual ya está acostumbrado a usar un smartphone son muy pocas, ya que podría realizarlas a través de su teléfono, en cambio, a una persona de tercera edad a la cual le cuesta hacer un uso normal de dispositivos tecnológicos, un dispositivo asistente inteligente podría hacerle la vida mucho más sencilla, ya que lo único que necesitaría sería establecer una conversación oral con el dispositivo, evitando el intermediario que supone un ordenador o un smartphone.

El rango de personas de tercera edad, es por tanto el público que más pueda necesitar un dispositivo asistente, como el público que más puede llegar a exprimirlo.

La creación de un asistente que pueda resolver sus dudas, con el que pueda hablar, al que pueda preguntar a qué hora es la partida de cartas en el bar, o al que puedan pedir auxilio en caso de una caída, puede ser de gran utilidad para mejorar su día a día, al igual que servir de alivio para el resto de familiares que no pueden estar cerca de sus seres queridos, sabiendo que van a poder ser informados rápidamente de un posible accidente.

He aquí, por tanto, el agujero de mercado que ha sido encontrado en las grandes empresas antes nombradas, y el cual puede permitir un modelo de negocio el cual tenga como premisa buscar la satisfacción de las personas mayores, que igualmente puede servir para dar libertad a otros grupos sociales, como pueden ser las personas con algún tipo de discapacidad.

El auge de los dispositivos asistentes se puede ver como una evolución tecnológica que permita facilitar las rutinas, pero también con ello se pone en vilo la gestión de la privacidad e intimidad que hay dentro de nuestras casas. Esto es debido a que este tipo de dispositivos inteligentes pueden recolectar información a través de escuchar las conversaciones de un usuario y su día a día, siendo información que no se sabe de manera exacta a dónde va a parar, o qué se va a hacer con ella.

Esta información captada puede ser de un caracter sensible, ya que puede contener desde simples gustos, incluyendo las necesidades del usuario, como sus tendencias políticas, siendo datos que pueden estar siendo utilizados por terceros, o incluso pudiendo estar siendo vendidos.

La falta de soluciones en el mercado que cumplan todos estos puntos supone una motivación para la creación de un asistente que no esté conectado constantemente a Internet, ya que las personas de edad avanzada seguramente no tengan contratado el servicio, y que tampoco almacene datos de carácter sensible de los usuarios, sino que simplemente interaccione con cada persona y la ayude en su día a día, tanto ofreciendo actividades de la misma localidad, como respondiendo cualquier pregunta que pase por la cabeza de quien lo posea, al igual que buscando ayuda en caso de que una persona requiera auxilio.

Antes de adentrar en la elección de un asistente inteligente, se expondrá qué es, y cuál es su funcionamiento, para entender mejor los motivos por los cuales se acaba eligiendo uno en vez de otro.

\subsubsection{Qué es un asistente inteligente}

Un asistente inteligente es una máquina programada de tal manera que su comportamiento se asemeje al de una persona a la que se solicita asistencia, como su propio nombre indica, pudiendo mantener una conversación que siga los protocolos de comunicación humana.

\begin{figure}[h!]
    \centering
    \includegraphics[width=10cm]{./img/sequence/human.png}
    \caption{Secuencia de Comunicación humana}
    \label{fig:humanseq}
\end{figure}

Como se puede ver en la figura \ref{fig:humanseq}, un protocolo de comunicación entre dos personas se basaría en un saludo para entablar conversación, para posteriormente realizar una pregunta.

En el caso de los asistentes, el proceso de conversación se basa en lo mismo: un usuario saluda al asistente mediante el uso de una palabra o conjunto de palabras, al que se llamará \textbf{hotword}, que cuando sea reconocido por el asistente inteligente, devolverá el saludo.
Es entonces cuando el usuario debe realizar la pregunta o solicitar la información que requiera.
Una vez hecha la pregunta, el dispositivo se pondrá a pensar la posible respuesta, entrando en el proceso al que se llamará \textbf{reconocimiento de los hechos}. Una vez identificados los hechos, devolverá la respuesta que más se acerque a lo deseado, gracias a un entrenamiento previo.

\subsubsection{Cómo piensa el Asistente}

El proceso de pensamiento analizado de los principales asistentes del mercado, que se expondrá en el presente capítulo, tiene una estructura similiar independientemente del tipo de asistente que se trate, asemejándose a la figura \ref{fig:humasseq}.

\begin{figure}[h!]
    \centering
    \includegraphics[width=13cm]{./img/sequence/humasseq.png}
    \caption{Secuencia de Comunicación humano-asistente}
    \label{fig:humasseq}
\end{figure}

Lo que diferencia a un asistente de otro, es la manera en la cual piensa la respuesta, dando mayor validez a un asistente que dé una respuesta más aproximada a lo solicitado.
Para lograr dar la respuesta más acertada, la mayoría de los asistentes existentes realizan el procesamiento de la respuesta en la nube debido a la gran cantidad de información que tienen previamente almacenada de otras consulta de otros usuarios para contrastar los hechos capturados, al igual que también pueden realizar en la nube el proceso de speech-to-text enviando los audios a sus servidores y transformandolo ahí a cadenas de texto, o el de text-to-speech, que sería el proceso inverso.


\section{Requisitos mínimos}
Una vez que ya se ha expuesto cual es el funcionamiento base de un asistente virtual inteligente, interesa saber cuál es de todos los existentes que más se adecúa a las necesidades establecidas para este proyecto, de manera que se le exigirá que cumpla el máximo de los siguientes puntos expuestos.

\subsubsection{Despliegue en Dispositivos}

El asistente seleccionado debe poder ser desplegado de manera gratuita en un dispositivo formado por una placa RasbperryPi, o similar.

Para ello, se requiere que el asistente disponga de una librería o de un modo de uso que se pueda implementar en un dispositivo pequeño y portátil, facilitando el cambio de una ubicación física por otra de los distintos posibles puntos que existen dentro de un hogar, manteniendo una instalación lo más simple y limpia posible, en la que solo sea requerida la localización de un enchufe activo que posea corriente eléctrica.

\subsubsection{Tratamiento de la Información}

El asistente debe seguir la Ley Orgánica de Protección de Datos~\cite{lopd} y el Reglamento Europeo de Protección de Datos~\cite{rgpd}, y no debe tener acceso a escuchar conversaciones ajenas violando la intimidad de los usuarios.

Cualquier vacío legal o conjunto de cláusulas de extensa longitud no podrá será aceptado, para poder asegurar la protección de los datos de los usuarios.

En caso de almacenar algún tipo de dato, debe poder ser público para el usuario en cuestión, o haber sido aceptado expresamente por el usuario a favor de un control de su seguridad.

\subsubsection{Conexión a Internet}

La orientación de un dispositivo asistente inteligente como este a personas de una edad avanzada debe tener en cuenta que es posible que la mayoría de sus usuarios potenciales no tengan una conexión a internet en sus respectivos hogares.

Esto provoca que la conexión del dispositivo a la red tiene que ser lo mínima posible, favoreciendo los asistentes en los cuales el proceso de pensamiento sea ejecutado dentro del propio dispositivo, evitando tener que hacer el cálculo e identificación en servidores de alojados en la nube.

Es decir, el dispositivo debe ser capaz de funcionar en el lugar más remoto posible y sin ninguna conexión a internet, simplemente tras ser conectado a una red eléctrica.

\subsubsection{Adicción de Nuevas Tareas}

El software del asistente inteligente debe tener la capacidad de añadir nuevas tareas fácilmente, de modo que se puedan añadir nuevas funciones tanto propias, como desarrolladas por la comunidad de Internet.

\section{Mercado}
\label{mercato}
En los siguientes apartados se mostrarán las diferentes opciones que ofrece ya el mercado para el uso de asistentes inteligentes, con el fin de comprobar la existencia de alguno que ya tenga establecidos los requisitos base ya descritos, o que permita la opción de poder establecerlos.

    \subsection{Google Assistant}
    El asistente inteligente de Google es el que está en el año 2020 a la cabeza de los asistentes virtuales~\cite{top-asistentes}, y esto es debido a que viene instalado en todos los dispositivos Android, ocupando estos dispositivos la mayor parte del mercado de la telefonía móvil. Esta posición privilegiada le proporciona un mayor entrenamiento, dando por resultado una elevada tasa de acierto que como consecuencia le atribuye mayor usabilidad.

Esta creación de Google tiene una gran aceptación con los diferentes aparatos existentes para la domótica del hogar, facilitando la implementación y realización de tareas:

- Una tarea es un conjunto de acciones que se llevan a cabo tras accionarse un evento que hace de interruptor, pudiendo ser el evento tanto un comando de voz, la pulsación de un interruptor o la activación de un sensor, entre otras opciones.

De esta manera, Google permite el control de la domótica del hogar, o la realización de diferentes acciones en nuestro día a día, pero requiere ser configurado por cada usuario, evitando por tanto la posibilidad de un despliegue común.

\subsubsection{Cómo Funciona}

El proceso de pensamiento que hace el asistente de Google está basado en la nube, pero no solo eso sino que Google hace en sus servidores también el proceso de Speech-to-Text.

De este modo, Google almacena todos los audios\cite{google-almacena} que se le mandan a través de su asistente, estudiándolos uno a uno y asegurando o corrigiendo sobre la respuesta que envió el asistente.

Este proceso de corregir o confirmar es el método que tiene Google de entrenamiento, de manera que en la próxima consulta sobre el mismo tema, el asistente pueda responder con mayor exactitud.

Queda claro que sobre la teoría es un buen plan de entrenamiento, y en un mundo ideal esto sería perfecto, pero esos audios también pueden contener parte de información privada, o pueden servir para espiar conversaciones privadas que un usuario puede no querer que estén almacenadas, ni que sean escuchadas por otra persona, aunque sea un propio trabajador.

En cuanto a la posibilidad de ser desplegado en otro dispositivo que no sea un smartphone o una tablet, Google nos lo pone fácil, ya que otorga una librería con la cuál facilita la instalación y uso en una gran variedad de dispositivos, cuyo requisito es que dispongan de conexión a Internet.


    
    \subsection{Amazon Echo}
    ~También conocido como Alexa, es la opción creada por Amazon que más fuerza está tomando en la sociedad para ser elegida como el asistente de voz inteligente que nos acompañe en nuestro día a día.

A pesar de todas las ventajas posibles similares a las que se han descrito para el asistente de Google, también comparte sus contras, como las referentes a la localización donde se procesan los audios y donde se efectúa el entrenamiento, por lo que es otra opción que va a ser rechazada.

Sin tener en cuenta la ubicación del proceso de pensamiento, Amazon asegura\cite{escandalo-amazon} que almacena todos los audios hasta que es el propio usuario quien decide borrarlos, pero aún con la solicitud expresa del usuario no pueden ser borrados completamente ya que han sido compartidos con terceros, que son quienes han desarrollado funciones específicas, llamadas Skills. Estos desarrolladores no solo tienen acceso a los audios sino que los poseen, pudiendo hacer una mala práctica con ellos.

Por tanto, aunque el usuario en cuestión pida a Amazon el borrado de estos ficheros, la compañía únicamente podría borrar los archivos que posee, permaneciendo todos los cuales un desarrollador de estas skills haya almacenado, sin permitir que el propio Amazon sea capaz de eliminarlos aunque tratase de hacerlo.
    
    \subsection{Mycroft}\label{Microft}
    Ante la falta de privacidad por parte de Google y Amazon, se abre la puerta a Mycroft, proyecto open source que busca como pilar la seguridad de los datos y la protección de la privacidad, de manera que asegura no almacenar ningún dato o información del usuario potencial.

Mycroft tiene la mayor comunidad para el desarrollo e implementación de asistentes, de modo que dispone de una gran cantidad de skills que añadir a nuestro dispositivo, posicionándose como opción principal para la elaboración del proyecto.

Este asistente virtual puede ser fácilmente implementado en dispositivos tales como Raspberry, cumpliendo los propósitos y requisitos de este proyecto.

En cuanto a la conexión a Internet, el equipo de Mycroft informa acerca de estar trabajando en una herramienta que permita desplegar el asistente ya entrenado dentro del dispositivo, evitando la conexión, pero de momento esa herramienta no está finalizada, por lo que cojea en ese aspecto y de momento, debería tener una conexión permanente.

En caso de que estén en lo cierto y estén trabajando en esa herramienta, este asistente ocuparía la primera opción como asistente a desplegar, pero en el momento actual en el cual este proyecto cobra vida, no es viable su implementación.


    
    \subsection{Snips AI}
    Snips Seeed es una  plataforma de inteligencia artificial para el desarrollo de dispositivos asistentes.

Esta plataforma nos permite crear nuestro propio asistente basándose en un entrenamiento previo con unos hechos predefinidos por el desarrollador.

En ese entrenamiento previo, se permite al asistente tomar lo aprendido de otros skills, que son fáciles de añadir.

Tras la implementación en el dispositivo físico, el asistente ya ha sido entrenado, por lo cual no necesitaría volver a estar conectado a internet, siendo este el mayor argumento a favor ya que es la carencia del resto de asistentes.

En cuanto a su uso, el dispositivo solamente se mantiene a la espera de poder recibir el \textit{hotword}. Este hotword puede ser modificado por cualquier otra palabra que se desee, lo cual beneficia al proyecto actual con la posibilidad de asignar una palabra más acorde y familiar a las personas a quienes va orientado.

El asistente, en cuanto a su configuración interna, sigue el protocolo de comunicación descrito en la figura \ref{fig:humasseq}, con una pequeña modificación:
Cuando el servidor ya ha entendido la pregunta o solicitud, genera unos hechos definidos en su entrenamiento en forma de objeto que envía a través de un puerto MQTT, y se queda a la espera de una respuesta.

Los otros skills, que han sido añadidos previamente en el entrenamiento, están levantados escuchando por el puerto, de manera que identifican los objetos que circulan por él y comprueban si ese hecho le corresponde, que de no ser así, simplemente lo ignoran.
Una vez que un skill captura un hecho que sí que le corresponde, comprueba los \textit{slots} que contiene dicho hecho para ver qué es lo que se está solicitando. Tras esto, genera una respuesta en forma de cadena de texto, y la envía por el puerto MQTT para que la recoja el \textit{skill-server}.

Una vez capturada por este, simplemente transcribe el mensaje a audio con su skill de text-to-speech, y se lo transmite al usuario a través del altavoz, finalizando la comunicación.

Como se puede ver, el tratamiento de la información que se le otorga al dispositivo es el que nosotros deseemos, ya que el asistente no envia nada al exterior. De esta manera, nosotros controlamos qué sucede con la información, siendo el otro punto requerido en la búsqueda de un asistente ideal.

Como extra, Snips Seeed es una plataforma gratuita que se basa en una colaboración de una gran comunidad de desarrolladores. Pese a ser esta comunidad de menor tamaño que la conseguida por la opción de Mycroft, hay una gran cantidad de aplicaciones disponibles como juegos, la consulta del tiempo o la consulta de noticias, preparadas para ser asignadas directamente a nuestro dispositivo.

Como se puede ver, SnipsSeeed otorga las herramientas necesarias para cumplir todos los requisitos estipulados anteriormente, siendo la plataforma que se pone a la cabeza como posible elección.

    
    \subsection{Elección Final}

La elección más acorde de entre todas las propuestas es el uso Snips AI, ya que se adapta a todos nuestros requisitos.

En cuanto al proyecto, va a ser necesario el desarrollo de un dispositivo que contenga este asistente, al igual que el diseño y desarrollo de un backend capaz de albergar la información relativa a cada dispositivo.

Para la gestión de estos dispositivos también será necesaria la implementación de una página web, desde la cual puedan ser los dispositivos configurados, manejados y controlados.

La implementación operativa de todo este sistema formaría un proyecto de gran embergadura, por lo que en los siguientes capitulos se documentará el diseño y desarrollo del backend y del frontend, mostrando el control y gestión real de un dispositivo el cual su rango de funciones será breve, de modo que se establezca la base y se explique como puedan ser añadidas nuevas funciones de una manera simple y sencilla.


\chapter{Plan de Desarrollo}\label{cap.desarrollo}
\section{Introducción}
\subsection{Diagrama de casos de uso}

\subsection{Definición de casos de uso}

\subsection{Matriz de aproximación con requisitos}

\section{Metodología}
\subsection{Diagrama de casos de uso}

\subsection{Definición de casos de uso}

\subsection{Matriz de aproximación con requisitos}

\section{Planificación}
\label{planificacion}

Una vez conocido el método de trabajo que se seguirá para el desarrollo del proyecto, se deben conocer los plazos en los cuales el proyecto debe ser finalizado y entregado.

La fecha de inicio del proyecto es, por tanto, \textit{el 15 de Octubre de 2019}, y la fecha límite en la cual el proyecto debe ser entregado data del \textit{29 de Febrero de 2020}.

\subsubsection{Disponibilidad del desarrollador}

El proyecto se debe ajustar a una extensión de 300 horas, como fué especificado en el cálculo de costes en el apartado \ref{duracion_proyecto} del presente documento.

En cuanto a la disponibilidad del proyecto, el desarrollador carece de suficiente tiempo  al día como para elaborar 8 horas díarias, asemejándose a un trabajo real, debido a tener que compaginar las 7 horas diarias de un trabajo externo, con las necesarias para la completitud de dos asignaturas de la Universidad, y el presente proyecto.

Tras una planificación semanal, el desarrollador es capaz de asignar una media de 24 horas semanales, lo que supondría poder finalizar el proyecto en:

\begin{center}
    \textit{24horas/semana / 6 dias laborables/semana = 4horas/dia }
    
    \textit{300horas / 4 horas/dia = 75 días }  

\end{center}

\newpage
\section{Diagramas de Gantt}

Al datarse de un proyecto basado en prototipos y dada la amplitud posible del mismo, se va a proceder a organizar el plan de trabajo en la secuencia del siguiente ciclo:
\begin{enumerate}
    \item \textit{Elicitación de requisitos} \newline
    Periodo en el cual se razona cuáles son los requisitos necesarios que se van a llevar a cabo en el protototipo a desarrollar. En es este periodo en el cual se realiza el diagrama de Gantt del prototipo indicado, planeando y estipulando los tiempos necesarios para llevar a cabo cada tarea.
    
    \item \textit{Desarrollo} \newline
    Esta etapa es en la cual se lleva a cabo la elaboración del prototipo siguiendo el plan de desarrollo estipulado en el punto anterior.
    
    \item \textit{Presentación del prototipo} \newline
    Reunión con el tutor para mostrar el prototipo y sus funcionalidades, con el fin de ver si se ha alcanzado lo esperado, al igual que para realizar un brainstorm conjunto con el fin de poder elicitar los requisitos del prototipo siguiente.
\end{enumerate}

La cuantía de prototipos elaborados será tanta como permitan las 300 horas de duración máxima del proyecto, enfocando cada prototipo en la completitud de un mínimo de requisitos con los cuales se busca un acercamiento a un prototipo final lo más aproximado posible a un producto final.

Debido a la condición del desarrollador que va a llevar a cabo el proyecto, se considera que cada día marcado en los siguientes diagramas corresponde con 4 horas de trabajo.

Para llevar a cabo la organización, se ha respetado un día libre a la semana, el cual coincide con el Domingo, al igual que se respetan los días festivos nacionales.

\textit{NOTA: Finalmente, tras la elaboración iterativa y ajuste de prototipos a las horas máximas marcadas se ha conseguido elaborar la cuantía de 4 prototipos, de los cuales se mostrarán sus diagramas de Gantt a continuación.}

\begin{sidewaysfigure}
\begin{figure}[H]
    \centering
    \includegraphics[width=19cm]{./img/grantt/p0.png}
    \caption{Diagrama de Gantt - Prototipo 0}
    \label{fig:grant.p0}
\end{figure}
\end{sidewaysfigure}

\begin{sidewaysfigure}
\begin{figure}[H]
    \centering
    \includegraphics[width=19cm]{./img/grantt/p1.png}
    \caption{Diagrama de Gantt - Prototipo 1}
    \label{fig:grant.p1}
\end{figure}
\end{sidewaysfigure}

\begin{sidewaysfigure}
\begin{figure}[H]
    \centering
    \includegraphics[width=19cm]{./img/grantt/p2.png}
    \caption{Diagrama de Gantt - Prototipo 2}
    \label{fig:grant.p2}
\end{figure}
\end{sidewaysfigure}

\begin{sidewaysfigure}
\begin{figure}[H]
    \centering
    \includegraphics[width=19cm]{./img/grantt/p3.png}
    \caption{Diagrama de Gantt - Prototipo 3}
    \label{fig:grant.p3}
\end{figure}
\end{sidewaysfigure}

\begin{sidewaysfigure}
\begin{figure}[H]
    \centering
    \includegraphics[width=19cm]{./img/grantt/p4.png}
    \caption{Diagrama de Gantt: Prototipo 4}
    \label{fig:grant.p3s}
\end{figure}
\end{sidewaysfigure}

\newpage
\section{Recursos y Herramientas}
Para una correcta implementación que se adecúe a los requisitos, al igual que para un futuro manteniemiento del proyecto, enumerarán a continuación las herramientas que han sido utilizadas durante su desarrollo, con el fin de poder replicar en un futuro el entorno de trabajo en un futuro, o poder entender mejor cómo ha sido implementado.

    \subsection{Kotlin}

Kotlin es un lenguaje de programación funcional desarrollado por el equipo ruso de JetBrains como una evolución a Java por excelencia, permitiendo una interoperabilidad entre ambos, utilizándose bajo la JVM.

La sintaxis de Kotlin lo hace más intuitivo y simple, llegando al mismo resultado que en Java pero en un menor número de líneas, ahorrando tiempo y espacio.

Pese a que su gran popularidad es por el desarrollo Android, ahora está creciendo en su utilización en el lado del servidor gracias al framework de Ktor.~\cite{ktor}

    \subsection{Ktor}

Es un framework que se aprovecha de la usabilidad de Kotlin para el desarrollo de clientes y servidores asíncronos en la implementación de sistemas conectados.

Este framework será añadido al proyecto a través de la instalación de sus dependencias en gradle a través del fichero \textit{./build.gradle}.

    \subsection{Koin DI}
    \label{Koin}

Koin es un inyector de dependencias compatible con Kotlin, siendo bastante util para poder aplicar el principio de inversión de dependencias en el desarrollo del backend.~\cite{koin}

La configuración de este inyector de dependencias es bastante simple, teniendo una sintaxis sencilla y útil en la que únicamente hay que utilizar Kotlin, evitando así los ficheros de configuración en XML que obligan a usar otros tipos de inyectores.

La utilización de este framework será util para poder aislar los casos de uso del sistema, de modo que no tengan dependencia de ninguna otra clase u servicio externo a los casos de uso, aumentando la escalabilidad del sistema y permitiendo una extensibilidad de los casos de uso en un futuro.

Para su utilización, se añaden las dependencias al fichero \textit{./build.gradle}.

Se implementará a través de la instalación de su módulo en el main de la aplicación.
\begin{lstlisting}
    // Declare Koin
    install(Koin) {
        modules(myModule)
    }
\end{lstlisting}

y en el módulo de Koin se indicarán los servicios que serán inyectados.
\begin{lstlisting}
// app.di.Module.kt

/**
 * This module has got the Koin configuration with the declarations of the needed instances.
 * This instances could be injected now, applying the dependency inversion principle.
 */
val myModule = module {
    single { ConfRepo() as ConfService }
    single { DeviceRepo() as DeviceService }
    single { LocationRepo() as LocationService }
    single { PeopleRepo() as PeopleService }
    single { TaskRepo() as TaskService }
    single { UserRepo() as UserService }
    single { IntentRepo() as IntentsService }
    single { TokenCtrl() as AuthService }
}
\end{lstlisting}


    \subsection{OAuth 2.0}
    \label{oauth20}
    
OAuth 2.0 es un protocolo que establece las pautas para una conexión segura entre un servidor y un cliente.
Este protocolo se basa en la implementación de un servidor de tokens frente al cual hay que estar identificado.~\cite{oauth2}

La secuencia correcta para la utilización de este protocolo consta de los siguientes pasos:

\begin{enumerate}
        \item El usuario manda sus credenciales al sistema.
        \item El sistema comprueba que sus credenciales son correctas.
        \item El sistema solicita al servidor OAuth un nuevo inicio de sesión para un cierto usuario con las ciertas características.
        \item El servidor OAuth provee al sistema un conjunto de token temporales: uno de acceso que utiliazará el usuario para identificarse, y otro de refresco para que recupere el de acceso en caso de que se le caduque.
        \item El sistema devuelve los tokens al usuario
\end{enumerate}
Una vez que el usuario ya dispone de los tokens, simplemente tiene que realizar las peticiones normales a los end-points~\cite{endpoint} del sistema pero introducciendo el token de acceso en la cabecera de autorización en cada llamada, de modo que el sistema lo toma, comprueba contra el servidor OAuth2.0 que sigue siendo un token válido, y en caso de serlo, deja realizar al usuario la petición.

Para la implementación de este protocolo utilizaremos un framework adaptado a la lógica de Ktor.~\cite{myndocs.oauth2}

Este framework será adaptado a las características de uso del sistema, estableciendo el método de comprobación de credenciales que se considere oportuno, pudiendo variar en caso de que las credenciales correspondan con las de un dispositivo, o con las de un administrador del sistema.

\begin{enumerate}
    \item Añadir sus dependencia al fichero \textit{./build.gradle}.
    \item La adaptación de sus clases de identificadores y tienda de tokens con el fin de poder hacer una comprobación de credenciales y autenticar los tokens como requiera el sistema. Para ello, se han modificado dos clases: \textbf{InMemoryTokenStoreCustom} y \textit{InMemoryIdentityCustom}, las cuales se albergan en \textit{./app/oauth}.
    \item Instalación del servidor OAuth2.0 en nuestra aplicación.
    Dentro de \textit{./app/Application.kt}, instalamos el servidor:

    \begin{lstlisting}
        // instance of tokenStore manage tokens
        tokenStore = InMemoryTokenStoreCustom.get()
        
        // Install and configure the OAuth2 server
        install(Oauth2ServerFeature) {
            identityService = InMemoryIdentityCustom()
            clientService = InMemoryClient()
                .client {
                    clientId = Credentials.OAUTH_CLIENTID.value
                    clientSecret = Credentials.OAUTH_CLIENTSECRET.value
                    // client uri
                    redirectUris = setOf("uri/to/login")
                    // set access token to half an hour
                    accessTokenConverter = UUIDAccessTokenConverter(60*30)
                    // set refresh token to one week
                    refreshTokenConverter = UUIDRefreshTokenConverter(60*60*24*7)
                    authorizedGrantTypes = setOf(
                        AuthorizedGrantType.AUTHORIZATION_CODE,
                        AuthorizedGrantType.PASSWORD,
                        AuthorizedGrantType.IMPLICIT,
                        AuthorizedGrantType.REFRESH_TOKEN
                    )
                }
            // set the custom store to have access -> Singleton Pattern
            tokenStore = InMemoryTokenStoreCustom.get()
        }
    \end{lstlisting}
\end{enumerate}



    \subsection{Exposed SQL}

Exposed SQL es un framework que simplifica el acceso a la base de datos evitando las consultas puras de SQL mediante una sintaxis más simple que tiene definida en su \textbr{DSL}.~\cite{exposedsql}

El método por el cual se ha accedido al uso de este framework es por su facilidad para parsear los resultados obtenidos en cada consulta, al igual que para limitar los posibles ataques por \textbr{inyección de SQL}.

    \subsection{JSoup}

JSoup es una librería que permite la obtención del código fuente HTML resultante de sitios web, dando las herramientas necesarias para la captura y parseo de los diferentes valores en función de las etiquetas, clases, y jerarquía de los distintos elementos HTML obtenidos.~\cite{jsoup}
    
    \subsection{PostgreSQL}
    \label{postgresql}
También conocido como postgres, es un sistema de gestión de bases de datos relacionales, por lo que permite una orientación a objectos en las relaciones de los elementos contenidos en sus diferentes tablas.
Postgres es libre y de código abierto, lo que nos permite su uso en el proyecto.

    \subsection{VueJS}
    \label{Vue}

VueJS es un framework JavaScript que permite la creación y desarrollo de diferentes aplicaciones web.
Entre sus ventajas está su reactividad, permitiendo cambiar la información mostrada en la web de manera dinámica, evitando tener que recargar la página. 

Otro aspecto por el cual elegir este framework está en su simplicidad para crear e importar componentes, al igual que para enviar la información entre ellos.

También cabe destacar su modularidad: VueJS parte de cero, de modo que toda función de la que queramos disponer, simplemente tendrá que ser importada, pudiendo adaptar cualquier librería JavaScript facilmente. Esto permite un mayor acceso a todas las librerias ya existentes, al igual que la creación de un proyecto con un menor peso final, al contener únicamente las librerias que son necesarias.

El aspecto más importante por el cual VueJS está siendo utilizado es su curva de aprendizaje, característica con la que todo el mundo concuerda: es más facil de aprender a utilizar que sus principales competidores, entre los cuales se encuentran Angular y React.~\cite{vue-curve}

    \subsection{Bootstrap-Vue}

Bootstrap-Vue es una adaptación de la librería de componentes Bootstrap que permite la utilización de sus componentes en VueJS de una manera más simple y rápida.~\cite{bootstrap-vue}

    \subsection{Leaflet}

Leaflet es una biblioteca JavaScript basada en OpenMaps, permitiendo la utilización e implementación de mapas en nuestras páginas web de una manera gratuita gracias a ser de código abierto.~\cite{leaflet}

Leaflet tiene un gran desarrollo por parte de la comunidad, lo que hace a esta librería más interesante para la implementación de diferentes funciones, como puede ser la adición de cualquier tipo de marca en el mapa, la geocodificación a través de los atributos de un punto exacto como puede ser su calle o código postal, y permitiendo también la geocodificación inversa, obteniendo esos datos de un punto exacto a partir de las coordenadas.

    \subsection{Material icons}

Es la librería de iconos gratuíta de Google, la cual sigue los estilos de diseño de Material Design, basada en un estilo limpio y simple.

    \subsection{Astah UML}
Es una herramienta cuya función es la creación de diagramas y esquemas UML, que nos será muy útil para realizar el diseño y organizar el desarrollo del sistema.



\chapter{Análisis del Proyecto}\label{cap.analisis}
\section{Introducción}
Los asistentes inteligentes están buscando su hueco en todo hogar, y esto es un hecho.
Estos asistentes nos hacen la vida más sencilla, ayudando a actualizar y controlar toda la domótica de nuestras casas con acciones que hace unos años solo podíamos imaginarlas en películas de ciencia ficción.

El concepto de asistentes de voz, que tan de moda está ahora entre la población joven, junto con la gente de edad avanzada que necesita interacción en sus vidas puede ser una combinación perfecta para ayudarles a no perder el contacto con la sociedad.

La creación de un asistente que pueda resolver sus dudas, con el que puedan hablar, al que puedan preguntar a qué hora es la partida de cartas en el bar, o al que puedan pedir auxilio en caso de una caída, puede ser de gran utilidad para mejorar su día a día, al igual que servir de alivio para el resto de familiares que no pueden estar cerca de sus seres queridos, sabiendo que van a poder ser informados rápidamente de un posible accidente.

Pero el auge de los dispositivos asistentes pone en vilo la gestión de la privacidad e intimidad que hay dentro de nuestras casas,n y todo esto es debido a que pueden recolectar información a través de escuchar las conversaciones de nuestro día a día, siendo información que no sabemos a dónde va a parar, qué se va a hacer con ella.

Esta información es de un caracter sensible, ya que puede contener desde nuestros simples gustos, incluyendo nuestras necesidades, hasta nuestras tendencias políticas, siendo datos que pueden estar siendo utilizados por terceros, o incluso siendo vendidos ante nuestro desconocimiento.

La falta de soluciones en el mercado que cumplan todos estos puntos suponen una motivación para la creación de un asistente que no esté conectado constantemente a internet, ya que las personas de edad avanzada seguramente no tengan contratado el servicio, y que tampoco almacene datos de carácter sensible de los usuarios, sino que simplemente interaccione con cada persona, y la ayude en su día a día, tanto ofreciendo actividades de la misma localidad, como respondiendo cualquier pregunta que pase por la cabeza de quien lo posea, al igual que buscando ayuda en caso de que una persona requiera auxilio.

Antes de adentrar en la elección de un asistente inteligente, se expondrá qué es, y cuál es su funcionamiento, para entender mejor los motivos por los cuales se acaba eligiendo uno en vez de otro.

\subsubsection{Qué es un asistente inteligente}

Un asistente inteligente es simplemente una máquina programada de manera que su comportamiento se asemeje al de una persona a la que se solicita asistencia, como su propio nombre indica, pudiendo mantener una conversación que siga los protocolos de comunicación humana.

\begin{figure}[h!]
    \centering
    \includegraphics[width=10cm]{./img/sequence/human.png}
    \caption{Secuencia de comunicación humana}
    \label{fig:humanseq}
\end{figure}

Como se puede ver en la figura \ref{fig:humanseq}, un protocolo de comunicación entre dos personas se basaría en un saludo para entablar conversación, para posteriormente realizar una pregunta.

En el caso de los asistentes, el proceso de conversación se basa en lo mismo: un usuario saluda al asistente mediante el uso de una palabra o conjunto de palabras, al que se llamará \textbf{hotword}, que cuando sea reconocido por el asistente inteligente, devolverá el saludo.
Es entonces cuando el usuario debe realizar la pregunta o solicitar la información que requiera.
Una vez hecha la pregunta, el dispositivo se pondrá a pensar la posible respuesta, entrando en el proceso al que se llamará \textbf{reconocimiento de los hechos}. Una vez identificados los hechos, devolverá la respuesta que más se acerque a lo deseado, gracias a un entrenamiento previo.

\subsubsection{Cómo piensa el asistente}

El proceso de pensamiento analizado entre los principales asistentes, que se expondrá en este capítulo tiene una estructura similiar independientemente del tipo de asistente que se trate, asemejándose a la figura \ref{fig:humasseq}.

\begin{figure}[h!]
    \centering
    \includegraphics[width=14cm]{./img/sequence/humasseq.png}
    \caption{Secuencia de comunicación humano-asistente}
    \label{fig:humasseq}
\end{figure}

Lo que diferencia a un asistente de otro, es la manera en la cual piensa la respuesta, dando mayor validez a un asistente que dé una respuesta más aproximada a lo solicitado.
Para esto, la mayoría de asistentes tienen el procesamiento de la respuesta en la nube debido a la gran cantidad de información que tienen almacenada para contrastar los hechos capturados, al igual que también tienen en la nube el proceso de speech-to-text o el de text-to-speech.


\section{Requisitos}

Para poder identificar qué es lo que se debe implementar, se debe realizar primero una identificación de las necesidades técnicas del proyecto a desarrollar, con el fin de identificar cuales son las implementaciones que se deben desarrollar.

\subsection{Requisitos Funcionales}

Los requisitos funcionales son un conjunto de funciones que debe poder servir el proyecto. Están basados en especificar qué es lo que el sistema debe ser capaz de hacer o permitir, y son los siguientes:

    \begin{enumerate}
    \item Los dispositivos deben ser capaces de identificarse ante el sistema.

    \item El sistema debe permitir obtener un nuevo token de acceso mediante un token de refresco.

    \item El sistema debe ser capaz de registrar nuevos dispositivos.

    \item El sistema debe devolver un par de tokens de autenticación a los dispositivos que se identifiquen.

    \item El dispositivo debe informar al sistema sobre su estado cada un cierto tiempo, que pueda ser configurable, siendo guardado en un fichero de configuración.

    \item El sistema debe poder actualizar remótamente el archivo de configuración tanto de dispositivos particulares, como de dispositivos de una misma localidad, como de todos los dispositivos en conjunto.

    \item El sistema debe poder asignar tareas a un dispositivo, a una localidad, o a todos los dispositivos.

    \item El sistema debe ser capaz de mandar tareas pendientes a realizar a un dispositivo, como puede ser actualizar su archivo de configuración, apagar el dispositivo, o reiniciarlo, cuando el dispositivo informe de que esté conectado.

    \item El dispositivo debe realizar las tareas por fecha de creación.

    \item El dispositivo debe informar al sistema cada vez que proceda a realizar una tarea.

    \item El sistema debe poder marcar tareas de un dispositivo específico como ya realizadas, guardando la fecha de realización.

    \item El sistema debe ser capaz de recibir la información básica sobre una conversación entre el asistente y el usuario, para monitorizar su uso.

    \item El dispositivo debe avisar al sistema sobre conversaciones que haya tenido con el usuario, mandando la información básica que confronte con los hechos.

    \item El sistema tiene que ser capaz devolver información básica sobre una localidad, como el número de teléfono, direcciones de correo, y noticias en función de la ubicación del dispositivo que llame al servicio.
    
    \item Un administrador debe poder iniciar sesión, identificandose a continuación mediante tokens.
    
    \item Un administrador debe ser capaz de ver todos los tipos de tareas disponibles.
    
    \item Un administrador debe poder crear nuevos tipos de tareas.
    
    \item El dispositivo debe tener un protocolo de adicción de nuevas tareas, de manera que al añadir una nueva tarea no haya posibilidad de afectar al resto ya implementadas.
    
    \item Un administrador debe poder asignar tareas a dispositivos especificos, a dispositivos de una localidad específica, o a todos los dispositivos en general.
    
    \item Un administrador debe poder ver las tareas pendientes de un dispositivo específico.
    
    \item Un administrador debe poder cambiar la configuración a dispositivos especificos, a dispositivos de una localidad específica, o a todos los dispositivos en general.
    
    \item Un administrador debe poder obtener una lista de todos los dispositivos, incluyendo estos sus últimos estados, últimas tareas realizadas, tareas pendientes, e información sobre el usuario asignado.
    
    \item Un administrador debe poder ver la actividad ocurrida en un intervalo específico de tiempo de un dispositivo sin asignar.
    
    \item Un administrador no debe poder ver actividad de un usuario que ya no tenga un dispositivo asociado.
    
    \item Un administrador debe poder ver la actividad de un dispositivo asignado ocurrida en un intervalo específico de tiempo cuya fecha mínima sea la fecha en que se estableció la asignación.
    
    \item Un administrador debe poder añadir nuevas localidades al sistema, incluyendo su nombre, coordenadas, y código postal.
    
    \item Un administrador debe poder obtener la lista de localidades almacenadas en el sistema, teniendo cada localidad información sobre los dispositivos asignados a clientes.
    
    \item Un administrador debe poder añadir nuevos clientes, que representarán a los poseedores del dispositivo. 
    
    \item Un administrador debe poder obtener la lista de clientes.
    
    \item Un administrador debe poder asignar un dispositivo a un cliente específico.
    
    \item Un administrador debe poder desasignar un cliente de un dispositivo.
    
    \item Un administrador debe ser capaz de poder ver cuales son las acciones que más se le han solicitado a un dispositivo, y las estadísticas de a qué horas ha sido más veces consultado.

\end{enumerate}




\subsection{Requisitos No Funcionales}

Los requisitos no funcionales, al contrario de los descritos en el apartado anterior, se basan en la descripción de cómo esta implementado el sistema y sobre cómo debe interacturar.

    \begin{enumerate}[label=NF\arabic* -]

    \item La base de datos debe ser una base de datos de tipo relacional, preferiblemente usando PostgreSQL.
    
    \item El backend debe estar implementado con un lenguaje que pueda ser ejecutado bajo la JVM.
    
    \item Las contraseñas que se almacenen en el sistema deben estar encriptadas mediante una encriptación de tipo SALT.
    
    \item El administrador debe poder ver la ubicación de los dipositivos a través de un mapa implementado con Leaflet y OpenMaps.
    
    \item De un cliente se debe almacenar su código postal, nombre, apellidos y número de documento nacional de identidad.
    
    \item En la web administrativa, un dispositivo SIN cliente asignado debe aparecer en el mapa ubicado en la costa atlántica, con el icono en color amarillo.
    
    \item En la web administrativa, un dispositivo CON cliente asignado debe aparecer en el mapa ubicado en la posición de su localidad, con el icono en color verde.
    
    \item En la web administrativa, todos los dispositivos de una localidad deben aparecer en el mapa agrupados, ocultando sus iconos y mostrando el número de ellos que hay en ese grupo.
    
    \item En la web administrativa, al hacer click en un grupo de dispositivos del mapa, se debe mostrar el icono de cada dispositivo en color verde.
    
    \item En la web administrativa, al hacer click en un icono del mapa, debe salir información sobre el cliente asignado, y un enlace a para ver sus estadísticas.
    
    \item El periodo de filtración de las estadísticas en la web administrativa debe ser de un día específico, o de un mes específico, o de un año específico.
    
    \item Las estadisiticas de acciones de un dispositivo deben ser mostradas en la web administrativa en un diagrama de tipo Doughnut, y las horas de uso en un diagrama de barras.
    
    \item La web administrativa debe tener un panel donde se vean los dispositivos mediante tarjetas, que cambien de color en función de tiempo que lleve el dispositivo sin ser usado.
    
    \item La web administrativa debe permitir filtrar las tarjetas en función de si los dispositivos están asignados, de su última fecha de uso, y de la última fecha de tarea realizada, mostrando cuánto tiempo hace de su último uso.
    
    \item La web administrativa debe permitir asignar tareas a los dispositivos de una manera rápida dando click a una tarjeta, al igual que dar acceso rápido al apartado de estadísticas y configuraciones, y mostrar información del usuario asignado.
    
\end{enumerate}

\newpage
\section{Riesgos}

        \subsection{Introducción}
        A continuación, se analizarán los posibles riesgos, siendo estos los posibles problemas futuros que puedan tener un efecto positivo o negativo en los objetivos del proyecto, incluyendo cada uno de ellos un conjunto de causa-efectos.
        
        Para tratar mejor un riesgo, hay que tener en cuenta dos aspectos: cuál es la probabilidad de que ocurra, y cuál es el impacto que tendría en el proyecto en caso de que ocurriese.
        Una vez estimadas estas dos variables, se puede asignar una posición de ese riesgo dentro de la matriz de la figura \ref{fig:riskmatrix}.
        
        Una vez asignada su posición, se deberá priorizar todo riesgo que ocupe una posición superior a la línea de tolerancia, posiciones que poseen un color mas oscuro en la Figura \ref{fig:riskmatrix}.
        
        La matriz de la figura \ref{fig:riskmatrix} ha sido rellenada con los riesgos identificados en el Apartado \ref{sec:riesgos.identificados}, mostrando por tanto, que en este proyecto no se ha detectado ningún riesgo con el que haya que tomar una priorización sobre el resto.
        
        
        \begin{figure}[H]
            \centering
            \includegraphics[width=12cm]{./img/spm/risk.matrix.png}
            \caption{Matriz de priorización de riesgos}
            \label{fig:riskmatrix}
        \end{figure}

\newpage
    \subsection{Riesgos identificados}
    \label{sec:riesgos.identificados}
        A continuación se clasifican los riesgos que pueden tener un impacto sobre el proyecto:
        %t1
        \begin{table}[H]
        \centering
        \begin{tabular}{|l|c}
        \hline
        \textbf{Riesgo}               & \multicolumn{1}{r|}{R-01}                                             \\ \hline
        \textbf{Descripción}          & \multicolumn{1}{X|}{El desarrollador puede enfermar debido a las  condiciones externas.}
        \\ \hline
        \textbf{Impacto}              & \multicolumn{1}{r|}{SIGNIFICANTE}                                             \\ \hline
        \textbf{Probabilidad}         & \multicolumn{1}{r|}{MODERADA}                                         \\ \hline
        \textbf{Plan de mitigación}   & \multicolumn{1}{X|}{Evitar acciones que atenten contra la salud. }
        \\ \hline
        \textbf{Plan de contingencia} & \multicolumn{1}{X|}{Trabajo desde casa.}
        \\ \hline
        \end{tabular}
        \caption{Riesgo 01 - Enfermedad}
        \label{table:riskill}
        \end{table}
        % t2
        \begin{table}[H]
        \centering
        \begin{tabular}{|l|c}
        \hline
        \textbf{Riesgo}               & \multicolumn{1}{r|}{R-02}                                             \\ \hline
        \textbf{Descripción}          & \multicolumn{1}{X|}{Mala planificación del proyecto.}
        \\ \hline
        \textbf{Impacto}              & \multicolumn{1}{r|}{SIGNIFICANTE}                                             \\ \hline
        \textbf{Probabilidad}         & \multicolumn{1}{r|}{MODERADA}                                         \\ \hline
        \textbf{Plan de mitigación}   & \multicolumn{1}{X|}{Reuniones cada dos semanas de seguimiento }
        \\ \hline
        \textbf{Plan de contingencia} & \multicolumn{1}{X|}{Aumentar el tiempo de trabajo hasta alcanzar lo planeado.}
        \\ \hline
        \end{tabular}
        \caption{Riesgo 02 - Planificación incorrecta}
        \label{table:malaplanif}
        \end{table}
        % t3
        \begin{table}[H]
        \centering
        \begin{tabular}{|l|c}
        \hline
        \textbf{Riesgo}               & \multicolumn{1}{r|}{R-03}                                             \\ \hline
        \textbf{Descripción}          & \multicolumn{1}{X|}{Pérdida del trabajo elaborado.}
        \\ \hline
        \textbf{Impacto}              & \multicolumn{1}{r|}{ALTO}                                             \\ \hline
        \textbf{Probabilidad}         & \multicolumn{1}{r|}{BAJA}                                         \\ \hline
        \textbf{Plan de mitigación}   & \multicolumn{1}{X|}{Utilización de varios servicios de control de versiones. }
        \\ \hline
        \textbf{Plan de contingencia} & \multicolumn{1}{X|}{Recuperar versiones más recientes.}
        \\ \hline
        \end{tabular}
        \caption{Riesgo 03 - Pérdida del trabajo}
        \label{table:riskperdida}
        \end{table}
        % t4
        \begin{table}[H]
        \centering
        \begin{tabular}{|l|c}
        \hline
        \textbf{Riesgo}               & \multicolumn{1}{r|}{R-04}                                             \\ \hline
        \textbf{Descripción}          & \multicolumn{1}{X|}{Caída de los servidores que alojan el servicio.}
        \\ \hline
        \textbf{Impacto}              & \multicolumn{1}{r|}{SIGNIFICANTE}
        \\ \hline
        \textbf{Probabilidad}         & \multicolumn{1}{r|}{BAJA}                                         \\ \hline
        \textbf{Plan de mitigación}   & \multicolumn{1}{X|}{ Usar servidores que aseguren un mínimo de disponibilidad. }
        \\ \hline
        \textbf{Plan de contingencia} & \multicolumn{1}{X|}{ Despliegue automático cuando el servidor arranque }
        \\ \hline
        \end{tabular}
        \caption{Riesgo 04 - Caída de los servidores}
        \label{table:riskperdida}
        \end{table}
        % t5
        \begin{table}[H]
        \centering
        \begin{tabular}{|l|c}
        \hline
        \textbf{Riesgo}               & \multicolumn{1}{r|}{R-05}                                             \\ \hline
        \textbf{Descripción}          & \multicolumn{1}{X|}{Rotura del equipo del dispositivo}
        \\ \hline
        \textbf{Impacto}              & \multicolumn{1}{r|}{MODERADO}                                             \\ \hline
        \textbf{Probabilidad}         & \multicolumn{1}{r|}{BAJA}                                         \\ \hline
        \textbf{Plan de mitigación}   & \multicolumn{1}{X|}{ Disponer de un mínimo de dos dispositivos. }
        \\ \hline
        \textbf{Plan de contingencia} & \multicolumn{1}{X|}{Comprar otro dispositivo.}
        \\ \hline
        \end{tabular}
        \caption{Riesgo 05 - Rotura de dispositivo}
        \label{table:riskrotura}
        \end{table}
        % t6
        \begin{table}[H]
        \centering
        \begin{tabular}{|l|c}
        \hline
        \textbf{Riesgo}               & \multicolumn{1}{r|}{R-06}                                             \\ \hline
        \textbf{Descripción}          & \multicolumn{1}{X|}{ Desconexión de la red WIFI del hogar. }
        \\ \hline
        \textbf{Impacto}              & \multicolumn{1}{r|}{BAJO}                                             \\ \hline
        \textbf{Probabilidad}         & \multicolumn{1}{r|}{SIGNIFICANTE}                                         \\ \hline
        \textbf{Plan de mitigación}   & \multicolumn{1}{X|}{ Conexión a la red a través de tarjeta SIM. }
        \\ \hline
        \textbf{Plan de contingencia} & \multicolumn{1}{X|}{ Recarga del saldo de la tarjeta SIM a través de Internet. }
        \\ \hline
        \end{tabular}
        \caption{Riesgo 06 - Desconexión de la red. }
        \label{table:riskdisconn}
        \end{table}
        % t7
        \begin{table}[H]
        \centering
        \begin{tabular}{|l|c}
        \hline
        \textbf{Riesgo}               & \multicolumn{1}{r|}{R-07}                                             \\ \hline
        \textbf{Descripción}          & \multicolumn{1}{X|}{Privatización de algún servicio.}
        \\ \hline
        \textbf{Impacto}              & \multicolumn{1}{r|}{MODERADO}                                             \\ \hline
        \textbf{Probabilidad}         & \multicolumn{1}{r|}{BAJA}                                         \\ \hline
        \textbf{Plan de mitigación}   & \multicolumn{1}{X|}{ Implementación del software adaptativa a cualquier servicio, para poder ser sustituído.
        
        Conocimiento del funcionamiento de otros servicios alternativos. }
        \\ \hline
        \textbf{Plan de contingencia} & \multicolumn{1}{X|}{ Sustitución del servicio.}
        \\ \hline
        \end{tabular}
        \caption{Riesgo 07 - Privatización de algún servicio. }
        \label{table:riskpriv}
        \end{table}


    \newpage
    \subsection{Riesgos acontecidos}
    
        Durante el transcurso del proyecto se pudo observar la aparición de ciertos riesgos:
        
        \begin{enumerate}
        
        \item \textbf{Riesgo 01 - Enfermedad}
        
        Descrito en la tabla \ref{table:riskill}, el desarrollador del sistema enfermó durante el último tramo del año.
        
        Este riesgo produjo un retraso de dos semanas en la elaboración del proyecto que el desarrollador tuvo que recuperar quitándose horas de sueño, ya que compartía este proyecto con otro trabajo externo.
        
        \item \textbf{Riesgo 02 - Mala planificación del proyecto}
        
        Descrito en la tabla \ref{table:malaplanif}, al utilizarse una planificación del proyecto mediante un modelo incremental, no se tuvo en cuenta el tiempo necesario para la escritura de la documentación formal, manteniendo una documentación informal a modo de cuaderno de bitácora donde se iban narrando los hechos.
        Este modelaje incremental hizo que se desarollasen más funciones de las que se esperaban en un principio, por lo que el periodo de escritura de la documentación aumentó notablemente, retrasando como consecuencia la finalización total del proyecto.
        
            \item \textbf{Riesgo 07 - Privatización de servicios}
            
        Descrito en la tabla \ref{table:riskpriv}, ocurrió cuando la plataforma de nuestro asistente inteligente fue comprada por Sonos. 
        
        A fecha de 31 de Enero de 2020, Snips Seeed privatizó sus servicios, bloqueando las funciones y el acceso a entrenamientos de nuestro asistente. Pese a tener analizado como posible riesgo la privatización de algún servicio, no se esperaba que la privatización fuese del asistente en general, ya que se analizó su modelo de negocio, donde se podía observar que tenía una fuerza para posicionarse como una de las primeras potencias en el mundo de los asistentes virtuales en los años venideros, ya que era el único asistente virtual capaz de trabajar sin una conexión fija a Internet. 
        
        Sonos vió también ese potencial y quiso hacerlo suyo, comprando la compañía para poseer la mejor baza de entre todos los asistentes con el fin de convertirse en los próximos años en uno de los líderes de este sector tecnológico.
        
        Gracias al plan de contingencia, y a la elaboración de un sistema que no dependa de ningun servicio específico, sino que todos los servicios implementen las funciones que el sistema requiera, el asistente podrá ser sustituido por otro de los nombrados anteriormente, posicionándose como mejor opción el asistente de Mycroft, descrito en el apartado \ref{Microft} del documento.
        
        
        \end{enumerate}

\newpage
\section{Costes}

    Para poder dar un valor monetario al proyecto hay que tener en cuenta sus costes, siendo tanto costes económicos como costes de desarrollo para poder ser realizado.

\begin{enumerate}
    \item \textbf{Duración estimada del proyecto} \label{duracion_proyecto}
    
El proyecto descrito en el presente documento es un proyecto relativo a un Trabajo de Fin de Grado (TFG), por lo tanto su estimación de horas de elaboración deberán corresponder con el coste de horas estimadas en cuanto al valor hora-credito de la carrera.

Según la guía docente de la Universidad de Valladolid, cada crédito tiene una estimación de trabajo de 25 horas, y un TFG tiene asignado una cuantía de 12 créditos.

Por tanto, el trabajo estimado para la elaboración de un TFG es:

\begin{center}
    \textit{12 créditos * 25horas/crédito = 300 horas}
\end{center}

    \item\textbf{ Coste de personal:}
    
El salario medio para un desarrollador junior en España que todavía no tiene el título está en \textit{18966euros/año}, y calculando que al año se hacen unas 1900 horas de trabajo de media, este desarrollador junior debería cobrar aproximadamente:

\begin{center}
    \textit{18966euros/1año * 1año/1900horas = 10euros/hora}
\end{center}

Teniendo en cuenta que la elaboración del proyecto tiene una estimación de 300 horas, el coste de personal será de:

\begin{center}
    \textit{10 euros/hora * 300 horas = 3000 euros}
\end{center}

    \item \textbf{ Coste del equipo: }
    
Para el desarrollo del proyecto se necesita la adquisición de dos dispositivos completos.

El coste del kit que proporciona la plataforma de Snips Seeed es de 115 euros por cada dispositivo, por lo que sería necesaria la inversión de:

\begin{center}
    \textit{115 euros/dispositivo * 2 dispositivos = 230 euros}
\end{center}

    \item \textbf{Coste de suministros}
    
    La duración del proyecto a desarrollar por el desarrollador, teniendo en cuenta que realice una jornada de 8 horas al día, será:
    
\begin{center}
    \textit{1900 horas/año / 12 meses/año = 160 horas/mes}
    
    \textit{300 horas/tfg / 160 horas/mes = 1,875 meses/TFG}
\end{center}

    Como el gasto en suministros son cuotas mensuales, se aproxima a 2 meses de trabajo completo la resolución del TFG.
    
    El coste medio de consumo de suministros contando el acceso a internet y el consumo de la luz suma una cuantía de 25 euros al mes. Por tanto, el coste de suministros, sería de:
\begin{center}
    \textit{2 meses/TFG * 25 euros/mes = 50 euros/TFG}
\end{center}
    
\end{enumerate}

Teniendo en cuenta, según el libro de SPM ***********, que un proyecto tiende a retrasarse entre un 8\% y un 10\%, el coste total deberá ser recalculado, siendo la suma de todos los costes:

\begin{center}
    \textit{3000 + 230 + 50 = 3280 euros}
    
    \textit{3280 euros * 10\%  = 3608 euros}
\end{center}

El coste para la elaboración del proyecto, será por tanto la cuantía de \textbf{3608 euros}.



\chapter{Diseño}
\label{cap.diseño}
\newpage
\section{Casos de Uso}

    \subsection{Diagrama de casos de uso}

\subsection{Definición de casos de uso}

\subsection{Matriz de aproximación con requisitos}

\newpage
\section{Diagramas de Secuencia}

    
\subsection{Introducción}

En esta sección se van a plasmar los diagramas de secuencia con una breve explicación de lo que está sucediendo con el fin de que narrando los hechos se comprenda de una manera más sencilla qué es lo que se está representando y facilitando en un futuro una reimplementación.

\subsection{Relativos a interacción con el Dispositivo}

\subsubsection{Flujo del Dispositivo}

En este diagrama se puede observar como fluye el dispositivo entre sus distintos estados.

La secuencia representada muestra como el dispositivo arranca directamente este servicio, el cual inicia sesión, o permanece intentándolo, para posteriormente enviar su estado.

Se muestra el flujo tando del caso de uso de \textit{inicio de sesión del dispositivo}, como del caso de uso de \textit{notificación de dispositivo conectado}, que están representados en los cuadros D01 y D02, respectivamente.

\begin{figure}[H]
    \centering
    \includegraphics[width=12cm]{./img/sequence/diagram/device.png}
    \caption{Diagrama de secuencia DS01 - Dispositivo}
    \label{fig:seq.device}
\end{figure}

\subsubsection{Petición de Inicio de sesión}

En el siguiente diagrama de secuencia, el \textit{DS02}, se muestra la continuación del diagrama de secuencia anterior, haciendo incapié a qué es lo que ocurre en el lado del servidor una vez el dispositivo intenta iniciar sesión.

Como se puede observar, a un endpoint del servicio REST desplegado le llega una petición de inicio de sesión: un flujo correcto muestra cómo dentro del servidor, es el \textit{DeviceService} quien comprueba los credenciales, que siendo correctos sigue el flujo de secuencia y se solicita unos nuevos tokens para un dispositivo concreto al servicio de autenticación.

Si observamos la capa de dominio del servidor, que llamaremos \textit{core}, está basada en la implementación de unas interfaces que están definidas en el propio core. De esta forma cualquer servicio que en un futuro se quiera usar deberá implementar estas interfaces. Así se asegura la integridad del sistema, permitiendo cambiar de servicio en un futuro sin necesidad de tocar el core, manteniendo la misma funcionabilidad del sistema.

\begin{figure}[H]
    \centering
    \includegraphics[width=14cm]{./img/sequence/diagram/login.png}
    \caption{Diagrama de secuencia DS02 - Login}
    \label{fig:seq.login}
\end{figure}

\subsubsection{Petición de Dipositivo conectado}

El diagrama de secuencia \textit{DS03}, representado en la figura \ref{fig:seq.alive}, muestra también el comportamiento del servidor tras otra petición.

En este caso se trata de una petición de aviso de que el dispositivo está conectado, documentada en el cuadro \textit{D02}, la cual se puede observar que tiene finalmente dos posibles finales correctos: la finalización, y la realización del caso de uso de \textit{realización de tarea pendiente}, el cual está plasmado en el caso de uso \textit{D03}.

En este diagrama se expone cómo el dispositivo realiza la petición, y tras esa petición, el servidor se encaga de comprobar primero si el token del dispositivo es correcto para posteriormente decirle al controlador de tareas, \textit{TaskCtrl}, que el dispositivo acaba de realizar la tarea de mostrar que está conectado.

El controlador de tareas almacena el nuevo estado del dispositivo para posteriormente recuperar todas las tareas pendientes, que en caso de no haber alguna, se responde al dipositivo con un código de estado HTTP cuyo valor es 200, finalizando el caso de uso.
En cambio, si el controlador sí que ha recibido tareas pendientes, se le responde al dispositivo un código 300, disparando un nuevo caso de uso en el dispositivo, el cual es la realización de tareas pendientes, como se ha nombrado anteriormente.

\begin{figure}[H]
    \centering
    \includegraphics[width=14cm]{./img/sequence/diagram/alive.png}
    \caption{Diagrama de secuencia DS03 - Alive}
    \label{fig:seq.alive}
\end{figure}

\subsubsection{Realización de Tareas pendientes}

El presente diagrama de secuencia muestra el un prototipo de flujo válido a implementar en los dispositivos para implementación del caso de uso de \textit{Realización de tarea pendiente D03}, siendo válido para cualquier tipo de tarea, que en este caso representará a la tarea configuración, la cual está asociada también al caso de uso de \textit{actualización de la configuración D04}, ya que esta tarea es un requisito indispensable del sistema, y de este modo se puede mostrar mejor la combinación de ambas dando como resultado la posibilidad de actualizar y controlar remotamente el dispositivo.

En esta representación gráfica de la secuencia por la cual se realiza el caso de uso, se observa que previamente el dispositivo ha recibido una respuesta del servidor a una \textit{petición de dispositivo conectado} la cual contiene un código de estado de código 300, la cual se comprueba y se obtiene del cuerpo del mensaje las tareas pendientes.

Una vez se tienen las tareas, se forma un comando, como se narra en el apartado \ref{seq:received300}, y es ejecutado contra el sistema.

Este comando hace que el sistema ejecute un script, \textit{init.sh}, el cual arranca la realización de la tarea, en este caso la de actualizar la configuración, la cual está implementada en \textit{Conf.py}.

La tarea primero avisa al servidor que procede a realizar la tarea pendiente, que si recibe un mensaje correcto \textit{-código 200-} procede a realizar la tarea solicitada.

Por tanto, la tarea de configuración solicitará al servidor el archivo de configuración, que en caso de recibirlo, actualizará el fichero.

Una vez actualizado, si todo ha sido correcto, avisa al servidor sobre su finalización, de manera que el servidor marcará la tarea como realizada.

En esta secuencia se puede observar un protocolo de actuación a la hora de tratar las tareas:

\begin{enumerate}
    \item Avisar que se va a hacer la tarea.
    \item Hacer la tarea.
    \item Avisar que se ha hecho la tarea.
\end{enumerate}

Gracias a este protocolo un administrador puede observar si una tarea no ha sido realizada porque no le ha llegado, o porque no ha sido capaz de realizarla, dando la posibilidad a una mejor gestión de los problemas.

\begin{figure}[H]
    \centering
    \includegraphics[width=14cm]{./img/sequence/diagram/conftask.png}
    \caption{Diagrama de secuencia DS04 - Realización de tareas pendientes}
    \label{fig:seq.alive}
\end{figure}

\subsubsection{Marcar Tarea como Realizada}

Tras comprobar en el servidor la validez del token con el cual se hace la petición y obtener el dispositivo asociado, se almacena un nuevo estado del dispositivo, para su visualización de a qué hora se ha realizado la acción y poder observar su actividad. Posteriormente se marca esa tarea como finalizada, asignándole la hora en la cual se ha recibido en el servidor.

\begin{figure}[H]
    \centering
    \includegraphics[width=14cm]{./img/sequence/diagram/doingTask.png}
    \caption{Diagrama de secuencia DS05 - Servidor : Marcar tarea como realizada}
    \label{fig:seq.doing}
\end{figure}


\subsubsection{Marcar la Configuración de un dispositivo como Actualizada}

La siguiente secuencia representa una de las acciones con mayor complejidad del sistema, ya que poder llevar el registro de cual es la configuración que posee cada dispositivo y a cual es a la que se ha actualizado, teniendo en cuenta que hay 3 posibles configuraciones paralelas, lleva un tedioso trabajo, que una vez clarificado no es difícil de entender.

Partimos de que el dispositivo informa que ha actualizado a la configuración que posee una marca de tiempo específica.

Entonces, el sistema obtiene las configuraciones globales, locales y propias que corresponden a cada dispositivo y comprueba:

Si el timestamp corresponde con la global, elimina todas las configuraciones globales, ya estén pendientes o no, asignadas a ese dispositivo, y añade la actualizada como no pendiente.

Si no corresponde con la global, si no que con la local, elimina todas las configuraciones locales asignadas a ese dispositivo, estén pendientes o no, y añade la actualizada como no pendiente.

Si tampoco corresponde con la local, si no que corresponde con la propia del dispositivo, elimina la configuración no pendiente asignada, y cambia el estado de la pendiente.

\begin{figure}[H]
    \centering
    \includegraphics[width=14cm]{./img/sequence/diagram/updateConf.png}
    \caption{Diagrama de secuencia DS06 - Servidor : Configuración actualizada}
    \label{fig:seq.UpdateConf}
\end{figure}

\newpage
\subsubsection{Obtención de la Configuración Actual}

Para obtener la configuración de un dispositivo, el sistema simplemente obtiene el identificador correspondiente al token de la cabecera, del cual recolecta los datos, como si tiene ese dispositivo un usuario asociado.

A partir de esos datos, recoge la configuración global, la local a partir de la asignación de usuario si posee, y la del propio dispositivo, tanto la pendiente como la instalada. De todas estas configuraciones se queda con la que tenga el timestamp más actual, de modo que será la que debe ser enviada.

Si la marca de tiempo de la que se va a devolver es más reciente que la que está guardada como la actual del dispositivo, se marca la que se devuelve como que tiene pendiente su instalación.
\newpage

\begin{figure}[H]
    \centering
    \includegraphics[width=14cm]{./img/sequence/diagram/getConfig.png}
    \caption{Diagrama de secuencia DS07 - Servidor : Obtener configuración actual}
    \label{fig:seq.GetCurrentConf}
\end{figure}

\subsection{Relativos a la Interacción de un Administrador}

\subsubsection{Creación de un nuevo Tipo de Tarea}

La creación de nuevas tareas en el dispositivo ha sido explicado previamente en el apartado \ref{seq:received300}, pero para poder mandar al dispositivo esa nueva tarea, hay que poder registrarla en el sistema.

En la siguiente secuencia se muestran los pasos por los cuales un nuevo tipo de tarea puede ser registrada, permitiendo posteriormente la asignación a los dispositivos.

En este caso, para diferenciar entre la creación de tipos de tareas y la creación de una asignación de tarea a los dispositivos, modificaremos el nombre de referencia, llamando a los tipos de tareas  \textit{eventos}, y a la asignación de dichos eventos a los dispositivos, \textit{tareas}.

\begin{figure}[H]
    \centering
    \includegraphics[width=14cm]{./img/sequence/diagram/AddEvent.png}
    \caption{Diagrama de secuencia DS08 - Creación de nuevo tipo de tarea}
    \label{fig:seq.NewEvent}
\end{figure}

\newpage
\subsubsection{Asignación de una Tarea a un Dispositivo}

Una vez se ha mostrado cómo puede ser la estructura del dispositivo para realizar las tareas, y se ha dado la posibilidad de crear eventos en el servidor, solo falta mostrar como son asignables esos eventos a los dispositivos generando nuevas tareas.

\begin{figure}[H]
    \centering
    \includegraphics[width=14cm]{./img/sequence/diagram/AssignTask.png}
    \caption{Diagrama de secuencia DS09 - Asignación de tarea a dispositivo}
    \label{fig:seq.AssignTask}
\end{figure}

\newpage
\subsubsection{Recolección de Eventos}
Como bien se ha definido en los requisitos y se ha estipulado en el caso de uso \textit{A04}, un administrador debe poder obtener la lista completa de eventos del sistema, con el fin de saber cuáles están ya definidos y pueden ser asignados a los dispositivos.

\begin{figure}[H]
    \centering
    \includegraphics[width=14cm]{./img/sequence/diagram/GetEvents.png}
    \caption{Diagrama de secuencia DS10 - Recolección de eventos}
    \label{fig:seq.GetEvents}
\end{figure}

\subsubsection{Obtención de Tareas de un Dispositivo}

En los requisitos funcionales se data la necesidad de un método que permita la obtención de todas las tareas que han sido asignadas a un dispositivo en un determinado rango de tiempo, por lo que en el siguiente diagrama de secuencia se muestra la implementación.

\begin{figure}[H]
    \centering
    \includegraphics[width=14cm]{./img/sequence/diagram/TareasDeDispositivo.png}
    \caption{Diagrama de secuencia DS11 - Obtención de tareas de un dispositivo}
    \label{fig:seq.getDeviceTasks}
\end{figure}

\newpage
\subsubsection{Cambiar Configuración}

El administrador es capaz de cambiar la configuración tanto global, como local o de un dispositivo específico a través de una llamada a un endpoint del servicio REST del sistema.
Lo que ocurre en el sistema una vez se realiza la petición es documentado en el siguiente diagrama de secuencia.

\begin{figure}[H]
    \centering
    \includegraphics[width=14cm]{./img/sequence/diagram/CambiarConfiguracion.png}
    \caption{Diagrama de secuencia DS12 - Cambiar configuración}
    \label{fig:seq.ChangeConf}
\end{figure}

\newpage
\subsubsection{Obtención de todos los Dispositivos}

Para poder llevar un registro del estado actual de los dispositivos y ver si alguno está dando algún tipo problema es necesario poder obtener todos los dispositivos.
El siguiente diagrama de secuencia muestra cómo el sistema recoge primero todos los dispositivos que contiene, para posteriormente añadir a cada uno de ellos información como cuáles han sido los últimos estados que ha enviado, o cuáles son las últimas 5 tareas que ha realizado. También, cuáles han sido los últimos \textit{intents} que ha realizado, representando un intent a la interación entre usuario y dispositivo, al igual que obtiene las tareas pendientes que tiene cada dispositivo.

\begin{figure}[H]
    \centering
    \includegraphics[width=14cm]{./img/sequence/diagram/ObtenerDispositivos.png}
    \caption{Diagrama de secuencia DS13 - Obtención de todos los dispositivos}
    \label{fig:seq.GetDevices}
\end{figure}

\newpage
\subsubsection{Obtención de Actividad}

Un requisito del sistema era la necesidad de un servicio que obtuviese la actividad relacionada con un dispositivo en un periodo establecido, pero también la implementación de un servicio que obtuviese la actividad de un determinado usuario.

Por motivos de seguridad, se ha implementado la obtención de la actividad de un dispositivo, de la cual se puede obtener la sucedida entre cualquier periodo de fechas siempre y cuando no tenga ningún usuario asociado.

En caso de que un dispositivo tenga un usuario asociado en el momento de la consulta, los resultados se limitarán a mostrar la actividad de ese dispositivo únicamente con el usuario actual.

De este modo, como se puede observar en el diagrama de secuencia incluido a continuación, se evita la posibilidad de que un usuario acceda a consultas de actividades hechas con el mismo dispositivo por otro usuario, al igual que no permite obtener a los administradores la actividad de un determinado usuario, si el usuario ya no tiene el dispositivo asignado, respetando por tanto la privacidad.

\begin{figure}[H]
    \centering
    \includegraphics[width=14cm]{./img/sequence/diagram/ObtenerActividad.png}
    \caption{Diagrama de secuencia DS14 - Obtención de actividad}
    \label{fig:seq.GetTask}
\end{figure}


\subsubsection{Adición de una Nueva Localidad}

Con el fin de una mejor organización del sistema, se debe poder añadir la localidad en la cual se esté operando. De este modo, los dispositivos y usuarios pueden estar asociados a una localidad específica.

\begin{figure}[H]
    \centering
    \includegraphics[width=14cm]{./img/sequence/diagram/NuevaLocalidad.png}
    \caption{Diagrama de secuencia DS15 - Adición de nueva localidad}
    \label{fig:seq.AddLocal}
\end{figure}

\subsubsection{Adición de un Nuevo Usuario}

Será necesario únicamente el nombre, documento nacional de identidad, y código postal.

\begin{figure}[H]
    \centering
    \includegraphics[width=14cm]{./img/sequence/diagram/NuevoUsuario.png}
    \caption{Diagrama de secuencia DS16 - Adición de nuevo usuario}
    \label{fig:seq.AddUser}
\end{figure}

\subsubsection{Asignación de Dispositivo}

Se creará únicamente una relación entre el usuario y el dispositivo, estando el dispositivo asociado a una determinada localidad a través del código postal de su poseedor.

\begin{figure}[H]
    \centering
    \includegraphics[width=14cm]{./img/sequence/diagram/NuevaRelacion.png}
    \caption{Diagrama de secuencia DS17 - Asignación de dispositivo}
    \label{fig:seq.AddUser}
\end{figure}

\subsubsection{Desasignación de Dispositivo}

Se debe permitir finalizar la relación entre un usuario y un dispositivo, de modo que el dispositivo pueda ser asociado a otro usuario en cuestión.

\begin{figure}[H]
    \centering
    \includegraphics[width=14cm]{./img/sequence/diagram/FinalizarRelacion.png}
    \caption{Diagrama de secuencia DS18 - Desasignación de dispositivo}
    \label{fig:seq.AddUser}
\end{figure}

\subsubsection{Obtención de una Relación de Usuario}

Por motivos de necesidad, el administrador puede requerir conocer cual es la relación de un dispositivo específico. Por ello, a continuación se muestra cómo el sistema consigue la información.


\begin{figure}[H]
    \centering
    \includegraphics[width=14cm]{./img/sequence/diagram/ObtenerRelacion.png}
    \caption{Diagrama de secuencia DS19 - Obtención de relación}
    \label{fig:seq.AddUser}
\end{figure}


\chapter{Implementación}\label{cap.implementation}
\section{Planteamiento}
Una buena realización de un proyecto comienza con un buen diseño inicial en el que se establecen las bases para un desarrollo correcto.

Este proyecto puede ser dividido en 3 subproyectos, donde cada subproyecto tendrá una arquitectura diferente debido a las necesidades que requiere su implementación.

El primer subproyecto es el asignado a la implementación de un software que será desplegado en el dispositivo y que permita la conexión con el servidor, aceptando la manipulación del propio dispositivo de manera remota.

El segundo subproyecto corresponderá con el propio sistema alojado en el servidor, al que llamaremos backend, el cual tendrá acceso a una base de datos y servirá una API REST para permitir una comunicación entre los dispositivos y los administradores.

El tercer subproyecto corresponde con el desarrollo de un sitio web, al que se llamará frontend, el cual permite el acceso a los administradores para el control y gestión de los dispositivos, al igual que para consultar las estadísticas.

\section{Arquitectura}
\subsection{Dispositivo}

En cuanto a la arquitectura final del dispositivo, no va a ser especificada ya que el presente proyecto no se adentra en la implementación del asistente, sino que pone las pautas y los protocolos mediante los cuales el asistente podrá comunicarse con el servidor.

Para poder mostrar estas pautas se ha implementado un controlador de pruebas, que será explicado junto a los casos de uso del dispositivo, pudiendo permitir posteriormente el desarrollo del asistente virtual inteligente con cualquiera de los métodos existentes.

\subsection{BackEnd} \label{arch-be}

Como se ha nombrado anteriormente, el sistema desarrollado en la parte del servidor deberá servir una API REST, de modo que la arquitectura elegida para la implementación del backend es una arquitectura REST, la cual se basa en ofrecer unos end-points desde los cuales se trata la información almacenada en el sistema.

Para una mejor implementación de esta arquitectura, se estructura el sistema bajo la Clean Architecture, de modo que se permite el desarrollo del sistema en una organización formada por capas, donde el acceso a la siguiente capa es lineal, evitando dependencias cruzadas que perjudiquen la escalabilidad del sistema.

\begin{figure}[h!]
    \centering
    \includegraphics[width=7cm]{./img/arch/cleanarch.png}
    \caption{The Clean Architecture}
    \label{fig:cleanarch}
\end{figure}

\subsection{FrontEnd}

El sitio web a desarrollar para la administración del sistema se contruye a partir del framework javascript de VueJS, como ya se ha mencionado anteriormente en el apartado \ref{Vue}.

VueJS propone una arquitectura MVVM~\cite{mvvm}, en la cual el diseño del sistema se basa en el desarrollo de múltiples componentes: Un componente es una unidad que dispone de su propio sistema de vista-presentador, teniendo una interfaz gráfica implementada en HTML + CSS, que efectúa un comportamiento a través de funciones JavaScript.

Esta modularidad interna basada en componentes debe seguir unas pautas~\cite{vuecomp} para sacar el máximo rendimiento de estos, al igual que para poder en un futuro remplazar o eliminar los componentes creados por otros que se ajusten a los nuevos requisitos sin perjudicar el resto de ellos.

Los datos que se manejan en la web aparecen en la vista de forma dinámica, de manera que es el presentador quien los puede variar.


\section{Implementación}
\subsection{Diagrama de casos de uso}

\subsection{Definición de casos de uso}

\subsection{Matriz de aproximación con requisitos}

%\newpage
%\section{Casos de uso}
%\subsection{Diagrama de casos de uso}

\subsection{Definición de casos de uso}

\subsection{Matriz de aproximación con requisitos}

%\newpage
%\section{Diagramas de secuencia}
%

\subsection{Introducción}

\subsection{Diagramas}

\begin{figure}[H]
    \centering
    \includegraphics[width=12cm]{./img/sequence/diagram/device.png}
    \caption{Diagrama de secuencia DS01 - Dispositivo}
    \label{fig:seq.device}
\end{figure}

\begin{figure}[H]
    \centering
    \includegraphics[width=14cm]{./img/sequence/diagram/login.png}
    \caption{Diagrama de secuencia DS02 - Login}
    \label{fig:seq.login}
\end{figure}

\begin{figure}[H]
    \centering
    \includegraphics[width=14cm]{./img/sequence/diagram/alive.png}
    \caption{Diagrama de secuencia DS03 - Alive}
    \label{fig:seq.alive}
\end{figure}

\begin{figure}[H]
    \centering
    \includegraphics[width=14cm]{./img/sequence/diagram/conftask.png}
    \caption{Diagrama de secuencia DS04 - Realización de tarea}
    \label{fig:seq.alive}
\end{figure}


        


\chapter{Despliegue}\label{cap.deploy}

En el presente capítulo se explicará los métodos por los cuales se debería poder desplegar todo el sistema de manera que podría dejarse una copia operativa que permitiese el control remoto de dispositivos asistentes.

\section{Introducción}

    Para que el sistema no presente fallos, una recomendación es el despliegue por orden de las pautas aquí mostradas, aun que no debería ser problema el despliegue en cualquier otro orden.

    La recomendación del orden a seguir en la guía es la siguiente:
    
    \begin{enumerate}
    
        \item Despliegue del backend.
        
        \item Despliegue del frontend.
        
        \item Instalación del dispositivo.
        
    \end{enumerate}
\newpage
\section{Guía}
\subsection{ Despliegue del Back End}
    Para el despliegue correcto del Back End es necesario seguir las siguientes pautas por el orden que marca la guía:

\subsubsection{Instalación de la base de datos}
    Para la base de datos se utilizará PostgreSQL, como se ha nombrado en la sección \ref{postgresql}.
    Esta base de datos puede ser sustituida por cualquier otra, como puede ser MySQL, pero deberá ser una base de datos relacional.
    
    Para ello, nos conectamos desde la máquina la cual hará de servidor, y seguimos los pasos:
    \begin{enumerate}
        \item Se obtienen los certificados:
            \begin{lstlisting}[language=bash]
        $ sudo apt-get install wget ca-certificates
            \end{lstlisting}
        
        \item Se añade la clave de postgresql:
            \begin{lstlisting}[language=bash]
        $ wget --quiet -O - https://www.postgresql.org/media/keys/ACCC4CF8.asc | sudo apt-key add -
            \end{lstlisting}
        
        \item Se configura:
            \begin{lstlisting}[language=bash]
        $ sudo sh -c 'echo "deb http://apt.postgresql.org/pub/repos/apt/ `lsb_release -cs`-pgdg main" >> /etc/apt/sources.list.d/pgdg.list'
            \end{lstlisting}
            
        \item Se actualiza:
            \begin{lstlisting}[language=bash]
        $ sudo apt-get update
            \end{lstlisting}
            
        \item Se instala postgresql:
            \begin{lstlisting}[language=bash]
        $ sudo apt-get install postgresql postgresql-contrib
            \end{lstlisting}
            
        \item Se crea la base de datos:
        
            Para ello, nos conectamos al usuario de postgres
            \begin{lstlisting}[language=bash]
        $ sudo su - postgres
            \end{lstlisting}
            Y accedemos a postreSQL:
            \begin{lstlisting}[language=bash]
        $ psql
            \end{lstlisting}
            
            Una vez dentro, se crea la base de datos, que en este caso se va a llamar \textit{assistant}, a la que podemos acceder tecleando simplemente su nombre.
            La creación de las tablas se va a dejar a Kotlin y a su framework Exposed, para evitar fallos en algún tipo de dato o una mala relación de claves.
            \begin{lstlisting}[language=bash]
        postgres@psql> CREATE DATABASE assistant;
            \end{lstlisting}
            
            Con esto sería suficiente, pero para proporcionar seguridad se procede al cambio de contraseña del usuario de postgres:
            \begin{lstlisting}[language=bash]
        postgres@psql> ALTER USER postgres PASSWORD 'nueva_contraseña';
            \end{lstlisting}
            
    \end{enumerate}
    
    Es muy importante recordar la contraseña, ya que es necesario utilizarla posteriormente en el despliegue del backend para permitir su acceso. Para el despliegue temporal, la contraseña utilizada será: \textbf{cu4lquie.Rar}
    
    En este punto, la base de datos ya estaría disponible para la conexión desde dentro del servidor, localizándola en el \textbf{puerto 5432}.

\subsubsection{Configuración y despliegue del backend}

    Para su despliegue de prueba, únicamente es necesaria la localización del archivo \textbf{JAR} y su colocación dentro del servidor en el directorio \textit{/home/assistant/core}. Este paso no es relevante, pero es recomendable para una correcta localización del archivo, ya que los logs del sistema se almacenarán y archivarán en una carpeta contenedora de esa ruta, de modo que todo se tendría mejor ordenado.
    
    Una vez con el fichero en esa ruta, tan solo se tiene que acceder a ese directorio y ejecutar el archivo:
    \begin{lstlisting}[language=bash]
        $ cd /home/assistant/core
        $ java -jar assistant.core.jar
    \end{lstlisting}
    
    Desplegado de esta manera se podría visualizar los logs del sistema en tiempo real, pero con el impedimento de que no se podría cerrar la consola. Para evitar este problema, se puede ejecutar de la siguiente manera, ocultando las salidas con el comando \textbf{nohup} y estableciéndolo en segundo plano con el símbolo \textbf{\&}
        \begin{lstlisting}[language=bash]
        $ cd /home/assistant/core
        $ nohup java -jar assistant.core.jar &
    \end{lstlisting}
    
    En caso de querer finalizar la ejecucción, tan solo habría que finalizar el proceso. Para ello obtenemos la lista de procesos:
    
    \begin{lstlisting}[language=bash]
        $ ps -ef | grep java
    \end{lstlisting}
    
    Lo que nos mostraría la lista de procesos java ejecutados en la máquina, de donde se puede ver el número de proceso que corresponde a nuestro backend, y se eliminaría de la siguiente forma:

    \begin{lstlisting}[language=bash]
        $ kill -9 numerodeproceso
    \end{lstlisting}
    
    El acceso al sistema desplegado de prueba podría darse a través de las credenciales de un administrador, las cuales tendrían un usuario llamado \textbf{admin}, al que le corresponde la contraseña \textbf{seren0314}. 
    
    En caso de querer cambiar la configuración por defecto, o los datos de acceso a la base de datos como el usuario y contraseña que se ha mencionado al comienzo de la guía, se deberá cambiar los valores establecidos del proyecto, que estan configurados en el siguiente fichero:
    
    \textit{/src/controller/model/util/Credentials.kt}
    
    Una vez estos datos hayan sido actualizados, se deberá volver a formar el archivo JAR, por lo que nos ayudaremos de la herramienta de gradle, que será ejecutada desde dentro de la carpeta del repositorio:
    
    \begin{lstlisting}[language=bash]
        $ gradlew shadowJar
    \end{lstlisting}
    
    Tras este comando, se generará \textbf{assistant.core.jar}, el cual está localizado en \textit{./build/lib/}.
    
    Una vez el archivo JAR está generado, se repetirían los pasos descritos al principio de la sección donde se ha descrito la posibilidad de un despliegue de prueba.
    
    Una vez desplegado, será accesible a través del puerto 8082.
    
    
\subsection{ Despliegue del Front End}
    Como ya se ha mencionado en numerosas ocasiones, el front end está elaborado con VueJS, el cual está integrado con el módulo de paquetes de Node, también conocido como \textbf{npm}.
Esto nos permite la compresión de la totalidad del proyecto en tan solo clases HTML, CSS y Javascript, además de los archivos de recursos como pueden ser las imágenes.

Para la realización de esta compresión tan solo hay que acceder a la carpeta del repositorio y ejecutar:

    \begin{lstlisting}[language=bash]
        $ npm run build
    \end{lstlisting}
    
    Lo cual generará una carpeta en el mismo repositorio llamada \textbf{dist}.
    
    El despliegue del front end se realizará por tanto copiando el contenido de esta carpeta y pegándolo dentro del servidor apache, que en el caso habitual se encuentra en la siguiente ruta:
    
    \textit{/var/www/html}
    
    y siendo accesible por tanto a través del puerto 80 de la ubicación de nuestro servidor.
    
\subsection{ Instalación del Dispositivo}
    La instalación del software en el dispositivo podrá ser flasheando una imagen con el sistema operativo y el asistente ya cargado.

La guía por tanto, no entrará en ese apartado ino en la posibilidad de instalar el asistente en cualquier dispositivo que disponga de una arquitectura Unix.

Para ello, se introducirá el repositorio \textbf{assistant.task} en el dispositivo, dentro del directorio \text y aprovechando los scripts contenidos en él, se instalará todo lo necesario.

\begin{enumerate}
    \item Acceso al repositorio:
    
        \begin{lstlisting}[language=bash]
            $ cd /home/pi/assistant.task
        \end{lstlisting}
    
    \item Instalacción de dependencias
    
        \begin{lstlisting}[language=bash]
            $ sh requirements.sh
        \end{lstlisting}
    
\end{enumerate}

Tras la ejecucción del script ya se instalan las dependencias requeridas y se configura el inicio automático del controlador remoto del asistante en cada reinicio de la máquina.



\newpage
\section{Trabajo futuro}
La elaboración de este proyecto deja abiertas las puertas a la elección e implementación final del asistente, ya que el presente documento narra la manera en la cual se ha implementado el controlador del dispositivo y cómo se puede manejar de manera remota un dispositivo, dejando sin implementar el asistente inteligente pero permitiendo la inclusión de cualquiera de los nombrados en el apartado \ref{mercato}, o cualquiera de los que serán creados en el futuro.

También, la manera de trabajar del controlador del dispositivo permite la inclusión sencilla de nuevas tareas, sin requerir la modificación de los archivos contenedores de las ya existentes.
 

%
En el presente capítulo se explicará los métodos por los cuales se debería poder desplegar todo el sistema de manera que podría dejarse una copia operativa que permitiese el control remoto de dispositivos asistentes.

\section{Introducción}

    Para que el sistema no presente fallos, una recomendación es el despliegue por orden de las pautas aquí mostradas, aun que no debería ser problema el despliegue en cualquier otro orden.

    La recomendación del orden a seguir en la guía es la siguiente:
    
    \begin{enumerate}
    
        \item Despliegue del backend.
        
        \item Despliegue del frontend.
        
        \item Instalación del dispositivo.
        
    \end{enumerate}
\newpage
\section{Guía}
\subsection{ Despliegue del Back End}
    Para el despliegue correcto del Back End es necesario seguir las siguientes pautas por el orden que marca la guía:

\subsubsection{Instalación de la base de datos}
    Para la base de datos se utilizará PostgreSQL, como se ha nombrado en la sección \ref{postgresql}.
    Esta base de datos puede ser sustituida por cualquier otra, como puede ser MySQL, pero deberá ser una base de datos relacional.
    
    Para ello, nos conectamos desde la máquina la cual hará de servidor, y seguimos los pasos:
    \begin{enumerate}
        \item Se obtienen los certificados:
            \begin{lstlisting}[language=bash]
        $ sudo apt-get install wget ca-certificates
            \end{lstlisting}
        
        \item Se añade la clave de postgresql:
            \begin{lstlisting}[language=bash]
        $ wget --quiet -O - https://www.postgresql.org/media/keys/ACCC4CF8.asc | sudo apt-key add -
            \end{lstlisting}
        
        \item Se configura:
            \begin{lstlisting}[language=bash]
        $ sudo sh -c 'echo "deb http://apt.postgresql.org/pub/repos/apt/ `lsb_release -cs`-pgdg main" >> /etc/apt/sources.list.d/pgdg.list'
            \end{lstlisting}
            
        \item Se actualiza:
            \begin{lstlisting}[language=bash]
        $ sudo apt-get update
            \end{lstlisting}
            
        \item Se instala postgresql:
            \begin{lstlisting}[language=bash]
        $ sudo apt-get install postgresql postgresql-contrib
            \end{lstlisting}
            
        \item Se crea la base de datos:
        
            Para ello, nos conectamos al usuario de postgres
            \begin{lstlisting}[language=bash]
        $ sudo su - postgres
            \end{lstlisting}
            Y accedemos a postreSQL:
            \begin{lstlisting}[language=bash]
        $ psql
            \end{lstlisting}
            
            Una vez dentro, se crea la base de datos, que en este caso se va a llamar \textit{assistant}, a la que podemos acceder tecleando simplemente su nombre.
            La creación de las tablas se va a dejar a Kotlin y a su framework Exposed, para evitar fallos en algún tipo de dato o una mala relación de claves.
            \begin{lstlisting}[language=bash]
        postgres@psql> CREATE DATABASE assistant;
            \end{lstlisting}
            
            Con esto sería suficiente, pero para proporcionar seguridad se procede al cambio de contraseña del usuario de postgres:
            \begin{lstlisting}[language=bash]
        postgres@psql> ALTER USER postgres PASSWORD 'nueva_contraseña';
            \end{lstlisting}
            
    \end{enumerate}
    
    Es muy importante recordar la contraseña, ya que es necesario utilizarla posteriormente en el despliegue del backend para permitir su acceso. Para el despliegue temporal, la contraseña utilizada será: \textbf{cu4lquie.Rar}
    
    En este punto, la base de datos ya estaría disponible para la conexión desde dentro del servidor, localizándola en el \textbf{puerto 5432}.

\subsubsection{Configuración y despliegue del backend}

    Para su despliegue de prueba, únicamente es necesaria la localización del archivo \textbf{JAR} y su colocación dentro del servidor en el directorio \textit{/home/assistant/core}. Este paso no es relevante, pero es recomendable para una correcta localización del archivo, ya que los logs del sistema se almacenarán y archivarán en una carpeta contenedora de esa ruta, de modo que todo se tendría mejor ordenado.
    
    Una vez con el fichero en esa ruta, tan solo se tiene que acceder a ese directorio y ejecutar el archivo:
    \begin{lstlisting}[language=bash]
        $ cd /home/assistant/core
        $ java -jar assistant.core.jar
    \end{lstlisting}
    
    Desplegado de esta manera se podría visualizar los logs del sistema en tiempo real, pero con el impedimento de que no se podría cerrar la consola. Para evitar este problema, se puede ejecutar de la siguiente manera, ocultando las salidas con el comando \textbf{nohup} y estableciéndolo en segundo plano con el símbolo \textbf{\&}
        \begin{lstlisting}[language=bash]
        $ cd /home/assistant/core
        $ nohup java -jar assistant.core.jar &
    \end{lstlisting}
    
    En caso de querer finalizar la ejecucción, tan solo habría que finalizar el proceso. Para ello obtenemos la lista de procesos:
    
    \begin{lstlisting}[language=bash]
        $ ps -ef | grep java
    \end{lstlisting}
    
    Lo que nos mostraría la lista de procesos java ejecutados en la máquina, de donde se puede ver el número de proceso que corresponde a nuestro backend, y se eliminaría de la siguiente forma:

    \begin{lstlisting}[language=bash]
        $ kill -9 numerodeproceso
    \end{lstlisting}
    
    El acceso al sistema desplegado de prueba podría darse a través de las credenciales de un administrador, las cuales tendrían un usuario llamado \textbf{admin}, al que le corresponde la contraseña \textbf{seren0314}. 
    
    En caso de querer cambiar la configuración por defecto, o los datos de acceso a la base de datos como el usuario y contraseña que se ha mencionado al comienzo de la guía, se deberá cambiar los valores establecidos del proyecto, que estan configurados en el siguiente fichero:
    
    \textit{/src/controller/model/util/Credentials.kt}
    
    Una vez estos datos hayan sido actualizados, se deberá volver a formar el archivo JAR, por lo que nos ayudaremos de la herramienta de gradle, que será ejecutada desde dentro de la carpeta del repositorio:
    
    \begin{lstlisting}[language=bash]
        $ gradlew shadowJar
    \end{lstlisting}
    
    Tras este comando, se generará \textbf{assistant.core.jar}, el cual está localizado en \textit{./build/lib/}.
    
    Una vez el archivo JAR está generado, se repetirían los pasos descritos al principio de la sección donde se ha descrito la posibilidad de un despliegue de prueba.
    
    Una vez desplegado, será accesible a través del puerto 8082.
    
    
\subsection{ Despliegue del Front End}
    Como ya se ha mencionado en numerosas ocasiones, el front end está elaborado con VueJS, el cual está integrado con el módulo de paquetes de Node, también conocido como \textbf{npm}.
Esto nos permite la compresión de la totalidad del proyecto en tan solo clases HTML, CSS y Javascript, además de los archivos de recursos como pueden ser las imágenes.

Para la realización de esta compresión tan solo hay que acceder a la carpeta del repositorio y ejecutar:

    \begin{lstlisting}[language=bash]
        $ npm run build
    \end{lstlisting}
    
    Lo cual generará una carpeta en el mismo repositorio llamada \textbf{dist}.
    
    El despliegue del front end se realizará por tanto copiando el contenido de esta carpeta y pegándolo dentro del servidor apache, que en el caso habitual se encuentra en la siguiente ruta:
    
    \textit{/var/www/html}
    
    y siendo accesible por tanto a través del puerto 80 de la ubicación de nuestro servidor.
    
\subsection{ Instalación del Dispositivo}
    La instalación del software en el dispositivo podrá ser flasheando una imagen con el sistema operativo y el asistente ya cargado.

La guía por tanto, no entrará en ese apartado ino en la posibilidad de instalar el asistente en cualquier dispositivo que disponga de una arquitectura Unix.

Para ello, se introducirá el repositorio \textbf{assistant.task} en el dispositivo, dentro del directorio \text y aprovechando los scripts contenidos en él, se instalará todo lo necesario.

\begin{enumerate}
    \item Acceso al repositorio:
    
        \begin{lstlisting}[language=bash]
            $ cd /home/pi/assistant.task
        \end{lstlisting}
    
    \item Instalacción de dependencias
    
        \begin{lstlisting}[language=bash]
            $ sh requirements.sh
        \end{lstlisting}
    
\end{enumerate}

Tras la ejecucción del script ya se instalan las dependencias requeridas y se configura el inicio automático del controlador remoto del asistante en cada reinicio de la máquina.



\newpage
\section{Trabajo futuro}
La elaboración de este proyecto deja abiertas las puertas a la elección e implementación final del asistente, ya que el presente documento narra la manera en la cual se ha implementado el controlador del dispositivo y cómo se puede manejar de manera remota un dispositivo, dejando sin implementar el asistente inteligente pero permitiendo la inclusión de cualquiera de los nombrados en el apartado \ref{mercato}, o cualquiera de los que serán creados en el futuro.

También, la manera de trabajar del controlador del dispositivo permite la inclusión sencilla de nuevas tareas, sin requerir la modificación de los archivos contenedores de las ya existentes.
 


%% escenarios

%\chapter{Referencias}\label{cap.referencias}
%\input{./secciones/s8referencias.tex}

%CóDIGO DE RECURSOS -> PASARLO A LA IMPLEMENTACIÓN
%IMPLEMENTACIÓN > DESPLIEGUE > PRUEBAS

\appendix

%\chapter{Manual de Usuario}\label{aped.A}
%\input{./anexos/a1manual.tex}
%\chapter{Contenidos del CD-ROM}\label{aped.B}
%\input{./anexos/a2cdrom.tex}

%Esto es una cita: \cite{ej}. Tiene que hacer referencia a la etiqueta de un bibitem.

%Esto es un enlace \href{www.enlace.net}{Enlace}


\cleardoublepage
\addcontentsline{toc}{chapter}{Bibliografía}
%\renewcommand\bibname{Referencias Web}
\begin{thebibliography}{X}

\bibitem{nielsen}
\textsc{The Nielsen Company} (2018). \textit{Nielsen: Despite their bast capabilities, smart speakers are all about the music}.
\\Recuperado a 17 de Enero de 2020, \\de \href{https://www.nielsen.com/us/en/insights/article/2018/smart-speaking-my-language-despite-their-vast-capabilities-smart-speakers-all-about-the-music/}

\bibitem{lopd}
\textsc{BOE.es} (2019, Junio 25). \textit{Ley Orgánica 3/2018, de 5 de diciembre, de Protección de Datos Personales y garantía de los derechos digitales}.
\\Recuperado a 17 de Enero de 2020, \\de \href{https://www.boe.es/buscar/act.php?id=BOE-A-2018-16673}

\bibitem{top-asistentes} 
\textsc{Laboratorio de Periodismo Luca de Tena} (2019, Octubre 17). \textit{El 42\% de los usuarios de altavoces inteligentes en España escuchan noticias a través de ellos}.
\\Recuperado a 19 de Enero de 2020, \\de \href{https://laboratoriodeperiodismo.org/altavoces-inteligentes-espana-noticias/}

\bibitem{google-almacena} 
\textsc{20Minutos} (2016, Junio 5). \textit{Google graba y almacena tu voz: dónde puedes escucharla (y borrarla)}.
\\Recuperado a 17 de Enero de 2020, \\de \href{https://www.20minutos.es/noticia/2762528/0/google-graba-almacena-tu-voz-donde-como-borrarlo/}

\bibitem{escandalo-amazon} 
\textsc{Píxel, El Mundo}. (2019, Julio 4) \textit{Escándalo en Amazon: la empresa reconoce que guarda tus conversaciones para siempre}.
\\Recuperado a 19 de Enero de 2020, \\de \href{https://www.elmundo.es/tecnologia/2019/07/04/5d1ccf42fc6c833f3f8b460d.html}

\bibitem{mycroft-doc}
\textsc{Mycroft AI} (n.d.). \textit{Documentation}.
\\Recuperado a 20 de Enero de 2020, \\de \href{https://mycroft-ai.gitbook.io/docs/}

\bibitem{mycroft-com}
\textsc{Mycroft AI} (n.d.). \textit{Community}.
\\Recuperado a 20 de Enero de 2020, \\de \href{https://community.mycroft.ai/}

\bibitem{mod-it-inc} 
\textsc{ProyectosAgiles.com} (n.d.). \textit{Desarrollo iterativo e incremental}.
\\Recuperado a 22 de Enero de 2020, \\de \href{https://proyectosagiles.org/desarrollo-iterativo-incremental/}

\bibitem{ktor} 
\textsc{JetBrains}, (n.d.). \textit{Ktor Servers}
\\Recuperado a 18 de Enero de 2020, \\de \href{https://ktor.io/servers/index.html}

\bibitem{koin} 
\textsc{InsertKoinIO} (2019, Noviembre 5). \textit{Koin}.
\\Recuperado a 18 de Enero de 2020, \\de \href{https://github.com/InsertKoinIO/koin#readme}

\bibitem{oauth2} 
\textsc{OAuth.net} (n.d.). \textit{OAuth 2.0}
\\Recuperado a 18 de Enero de 2020, \\de \href{https://oauth.net/2/}

\bibitem{endpoint} 
\textsc{SmartBear.com} (n.d.). \textit{What is an API Endpoint?}
\\Recuperado a 18 de Enero de 2020, \\de \href{https://smartbear.com/learn/performance-monitoring/api-endpoints/}

\bibitem{myndocs.oauth2} 
\textsc{Myndocs} (2019, Septiembre 7). \textit{Kotlin OAuth2 server}.
\\Recuperado a 18 de Enero de 2020, \\de \href{https://github.com/myndocs/kotlin-oauth2-server}

\bibitem{exposedsql} 
\textsc{JetBrains} (2019, Junio 3). \textit{Exposed SQL DSL}.
\\Recuperado a 18 de Enero de 2020, \\de \href{https://github.com/JetBrains/Exposed}

\bibitem{jsoup} 
\textsc{Jonathan H.} (n.d.). \textit{Download and install JSoup}.
\\Recuperado a 19 de Enero de 2020, \\de \href{https://jsoup.org/download}

\bibitem{vue-curve}
\textsc{Ben R.} (2019, Diciembre 5). \textit{Vue vs React: Which is the Better Framework?}
\\Recuperado a 19 de Enero de 2020, \\de \href{https://buttercms.com/blog/vue-vs-react-which-is-the-better-framework}

\bibitem{bootstrap-vue} 
\textsc{Bootstrap-Vue} (n.d.). \textit{Components}.
\\Recuperado a 19 de Enero de 2020, \\de \href{https://bootstrap-vue.org/docs/components}

\bibitem{leaflet} 
\textsc{Leaflet}(n.d.). \textit{An open-source JavaScript library for mobile-friendly interactive maps}.
\\Recuperado a 19 de Enero de 2020, \\de \href{https://leafletjs.com/reference-1.6.0.html}

\bibitem{hxc} 
\textsc{Universidad de Valladolid} (n.d.). \textit{Preguntas Frecuentes}.
\\Recuperado a 21 de Enero de 2020, \\de \href{http://www.uva.es/export/sites/uva/2.docencia/2.02.mastersoficiales/2.02.13.preguntasfrecuentes/index.html}

\bibitem{salario-junior} 
\textsc{Indeed.com} (2019, Diciembre). \textit{Salarios para empleos de Programador/a junior en España}.
\\Recuperado a 21 de Enero de 2020, \\de \href{https://es.indeed.com/salaries/programador-junior-Salaries}

\bibitem{spm}  \\
\textsc{Bob Hughes and Mike Cotterell} (2009). \textit{Software Project Management}. McGraw-Hill Education.

\bibitem{mvvm} 
\textsc{Jeremy L.} (2014, Abril 24). \textit{Model-View-ViewModel (MVVM) Explained}.
\\Recuperado a 23 de Enero de 2020, \\de \href{https://www.wintellect.com/model-view-viewmodel-mvvm-explained/}

\bibitem{vuecomp} 
\textsc{Eder N.} (2017, Octubre 9). \textit{The Vue architecture that worked for me. (in small and large apps)}.
\\Recuperado a 23 de Enero de 2020, \\de \href{https://medium.com/@ederng/the-vue-architecture-that-worked-for-me-in-small-and-large-apps-9b253cf92951}

\bibitem{crontab} 
\textsc{Vivek G.} (2019, Julio 19). \textit{Linux Execute Cron Job After System Reboot}.
\\Recuperado a 24 de Enero de 2020, \\de \href{https://www.cyberciti.biz/faq/linux-execute-cron-job-after-system-reboot/}

\bibitem{clean-arch-book}
\textsc{Robert C. Martin} (2017). \textit{Clean Architecture: A Craftsman's Guide to Software Structure and Design}. Pearson.

\end{thebibliography}


\end{document}
