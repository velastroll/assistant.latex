\documentclass[openright,twoside,10pt]{book}
\usepackage[b5paper,left=2cm,top=2.5cm,right=1.5cm,bottom=2.5cm]{geometry} 
\usepackage[spanish]{babel} % espanol
\usepackage[utf8]{inputenc} % acentos sin codigo
\usepackage{graphicx} % gráficos
\usepackage{lscape}
\usepackage{fancyvrb}
\usepackage{fancyhdr}
\usepackage{wrapfig}
\setlength{\parskip}{10pt plus 1pt minus 1pt}
 % aqui definimos el encabezado de las paginas pares e impares.
\rhead[]{}

\renewcommand{\headrulewidth}{0.5pt}

% aqui definimos el pie de pagina de las paginas pares e impares.
\rfoot[\thepage]{\thepage}
\cfoot[]{}
\renewcommand{\footrulewidth}{0pt}

%redefino el verbatim
%\renewenvironment{verbatim}{\begin{Verbatim}[frame=single,fontsize=\small]}{\end{Verbatim}}


% aqui definimos el encabezado y pie de pagina de la pagina inicial de un capitulo.
\fancypagestyle{plain}{
\fancyhead[R]{}
\fancyfoot[C]{}
\fancyfoot[R]{\thepage}
\renewcommand{\headrulewidth}{0.5pt}
\renewcommand{\footrulewidth}{0pt}
}

\pagestyle{fancy} % seleccionamos un estilo



\date{21 de Enero de 2020}
\author{Velasco Gil, Álvaro}

\title{TFG: titulo}

\begin{document}

\begin{titlepage}

\begin{center}
\vspace*{-1in}
\begin{figure}[htb]
\begin{center}
\includegraphics[width=3cm]{./img/logo}
\end{center}
\end{figure}
\begin{large}
\textbf{Universidad de Valladolid}
\end{large}

\vspace*{0.15in}

\vspace*{0.6in}
\begin{large}
\textbf{ESCUELA DE INGENIERÍA INFORMÁTICA}

\end{large}
\vspace*{0.2in}
\textbf{ GRADO EN INGENIERÍA INFORMÁTICA}\\
\textbf{ MENCIÓN EN INGENIERÍA DEL SOFTWARE }
\vspace*{0.1in}
\rule{140mm}{0.1mm}\\
\vspace*{0.2in}
\begin{large}
\textbf{{\LARGE DISEÑO E IMPLEMENTACIÓN DE UN SISTEMA DE CONTROL Y MONITORIZACIÓN DE ASISTENTES POR VOZ\\}}
\end{large}
\vspace*{0.2in}
\rule{140mm}{0.1mm}\\
\vspace*{2in}
\begin{large}
\begin{flushright}
\textbf{Alumno: Velasco Gil, Álvaro \\
\vspace*{0.3in}
Tutor: Vegas, Jesús }
\end{flushright}
\end{large}
\end{center}

\end{titlepage}

\newpage
\mbox{}	
\thispagestyle{empty} % para que no se numere esta página

\chapter*{}
\pagenumbering{Roman} % para comenzar la numeración de paginas en números romanos
\begin{flushright}
\textit{%Dedicatoria,\\
Pequeña dedicatoria.}
\end{flushright}

\chapter*{Agradecimientos} % si no queremos que añada la palabra "Capitulo"
\addcontentsline{toc}{chapter}{Agradecimientos} % si queremos que aparezca en el índice
\markboth{AGRADECIMIENTOS}{AGRADECIMIENTOS} % encabezado 

% Aquí agradecer

\chapter*{Resumen} % si no queremos que añada la palabra "Capitulo"
\addcontentsline{toc}{chapter}{Resumen} % si queremos que aparezca en el índice
\markboth{RESUMEN}{RESUMEN} % encabezado
\begin{flushleft}

El fin de este proyecto es el de crear un sistema capaz de monitorizar y controlar remotamente diferentes dispositivos que serán utilizados como asistentes inteligentes.

Para el desarrollo de este proyecto se ha utilizado diferentes lenguajes, como Kotlin en el backend, Python en los dispositivos, y VueJS, HTML y CSS en el frontend, siguiendo un plan basado en iteraciones.

% Aquí el resumen

\end{flushleft}

\chapter*{Abstract} % si no queremos que añada la palabra "Capitulo"
\addcontentsline{toc}{chapter}{Abstract} % si queremos que aparezca en el índice
\markboth{ABSTRACT}{ABSTRACT} % encabezado
\begin{flushleft}

% Aquí va el abstract

\end{flushleft}

\tableofcontents % indice de contenidos

\cleardoublepage
\addcontentsline{toc}{chapter}{Lista de figuras} % para que aparezca en el indice de contenidos
\listoffigures % indice de figuras

Aquí aparecerá la lista de imagenes que hay en el TFG

\cleardoublepage
\addcontentsline{toc}{chapter}{Lista de tablas} % para que aparezca en el indice de contenidos
\listoftables % indice de tablas

Aquí aparecerá la lista de tablas que hay en el TFG

\chapter{Introducción}\label{cap.introduccion}
\pagenumbering{arabic} % para empezar la numeración con números
\section{Introducción}

En este proyecto se abordará el proceso de control y monitorización remota de dispositivos asistentes inteligentes.

Tras la lectura de este documento se comprenderán tanto los motivos por los cuales se ha decidido tomar esta opción, como el proceso de despliegue del sistema, pasando por las fases de implementación en las que se enseñará a replicar proyectos de estructura similiar, como por las fases de desarrollo en la cuales se estudian los posibles riesgos y carácterísticas del proyecto, al igual que por el plan de desarrollo en el que se programa toda la elaboración.

\section{Motivación}

Los asistentes inteligentes están en auge y las grandes empresas están dando acceso a estos servicios de manera gratuita, donde lo único que hay que hacer para disfrutar de ellos es pagar es el dispositivo físico, en caso de que se requiera, lo que hace cuestionar la idea del modelo de negocio que están siguiendo para que salga rentable.

Este modelo de negocio hace pensar que el producto en realidad es cada usuario que lo utiliza, del que están recopilando información. Esta información al ser obtenida directamente de los hogares de cada usuario no es solamente privada, si no que también es íntima, lo que hace cuestionarse si realmente la comunidad de usuarios sabe que hay empresas aprovechándose de su día a día para recolectar toda la información mediante esos dispositivos que, situados en los lugares más personales de cada usuario, capturan información con gran potencial, ya sea sobre gustos, tendencias, necesidades o inclinaciones políticas, pudiendo todos estos datos estar siendo vendidos, o utilizados por terceros.

Esta ignorancia global sobre el tratamiento de los datos obtenidos por el dispositivo hace replantear la posibilidad de creacción de un nuevo dispositivo asistente, el cual haga más facil la vida de un sesgo de la sociedad al cual iría orientado, sin tener la necesidad ni posibilidad de almacenar datos de carácter sensible.

Por otro lado, la despoblación y migración por la parte joven de la sociedad está provocando una despobación en sus lugares de origen, dejando a los familiares de mayor edad en soledad, siendo la parte de la familia que a grosso modo necesita más atención y requiere más ayuda para pasar el día.

La implementación de un asistente inteligente el cual ayude a este conjunto de la sociedad a entretenerse proponiendo tanto eventos cercanos a ellos, como respondiendo sus dudas de una manera rápida, o sirviéndoles para buscar ayuda en caso de posible caída o solicitud de auxilio, puede ser una gran herramienta que mejore sus calidades de vida.


\section{Objetivos}

El objetivo de este proyecto es, por tanto,xs el despliegue de un sistema que permita la conexión de dispositivos asistentes al servidor de manera automática y remota, pudiendo ser controlados desde una web de administración, al igual que pudiendo ver estadísticas de uso para comprobar el estado de los usuarios. 

Estos dispositivos estarán orientados a personas de la tercera edad, estando asignado cada dispositivo a una persona, y conteniendo entonces información relacionada con la localidad a la hora de responder.

Los dispositivos podrán entonces ser controlados y configurados de manera individual, en función de su localidad, o de una manera global, pudiendo enviar actualizaciones a través de configuraciones, al igual que mandar la realización de diferentes acciones.


\chapter{Plan de desarrollo}\label{cap.plandes}
\section{Introducción}

\section{Metodología}

\section{Recursos y herramientas}
    \subsection{Kotlin}

Kotlin es un lenguaje de programación funcional desarrollado por el equipo ruso de JetBrains como una evolución a Java por excelencia, permitiendo una interoperabilidad entre ambos, utilizándose bajo la JVM.

La sintaxis de Kotlin lo hace más intuitivo y simple, llegando al mismo resultado que en Java pero en un menor número de líneas, ahorrando tiempo y espacio.

Pese a que su gran popularidad es por el desarrollo Android, ahora está creciendo en su utilización en el lado del servidor gracias al framework de Ktor. 

    \subsection{Ktor}

Es un framework que se aprovecha de la usabilidad de Kotlin para el desarrollo de clientes y servidores asíncronos en la implementación de sistemas conectados.

    \subsection{Koin DI}

Koin es un inyectorr de dependencias compatible con Kotlin, siendo bastante util para poder aplicar el principio de inversión de dependencias en el desarrollo del backend.    

    \subsection{OAuth 2.0}
    
    \subsection{Exposed SQL}
    \subsection{JSoup}
    \subsection{Postgres}
    \subsection{VueJS}
    \subsection{Bootstrap-vue}
    \subsection{Leaflet}
    \subsection{Material icons}

\section{Planificacion}

\chapter{Estado del arte}\label{cap.arte}
\section{Cómo andan las cosas}



\chapter{Diseño}
\label{cap.desarrollo}
\subsection{Diagrama de casos de uso}

\subsection{Definición de casos de uso}

\subsection{Matriz de aproximación con requisitos}

\chapter{Demo}\label{cap.demo}

\section{Introducción}
\subsection{Una subsección}


\chapter{Conclusión}\label{cap.conclusion}
\section{Introducción}

\section{Trabajo futuro}
 


\chapter{Referencias}\label{cap.referencias}
\section{Introducción}
\subsection{Una subsección}




\appendix

\chapter{Manual de usuario}\label{aped.A}
\input{./anexos/a1manual.tex}
\chapter{Contenidos del CD-ROM}\label{aped.B}
\input{./anexos/a2cdrom.tex}

%Esto es una cita: \cite{ej}. Tiene que hacer referencia a la etiqueta de un bibitem.

%Esto es un enlace \href{www.enlace.net}{Enlace}


\cleardoublepage
\addcontentsline{toc}{chapter}{Bibliografía}
%\renewcommand\bibname{Referencias Web}
\begin{thebibliography}{X}

\bibitem{ref1} \textit{Ejemplo}, \\
\textsc{ejemplo.com}.
\\Recuperado a tal fecha, \\de \href{http://ejemplo.com}


\end{thebibliography}


\end{document}
